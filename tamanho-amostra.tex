% Options for packages loaded elsewhere
\PassOptionsToPackage{unicode}{hyperref}
\PassOptionsToPackage{hyphens}{url}
%
\documentclass[
]{article}
\usepackage{amsmath,amssymb}
\usepackage{iftex}
\ifPDFTeX
  \usepackage[T1]{fontenc}
  \usepackage[utf8]{inputenc}
  \usepackage{textcomp} % provide euro and other symbols
\else % if luatex or xetex
  \usepackage{unicode-math} % this also loads fontspec
  \defaultfontfeatures{Scale=MatchLowercase}
  \defaultfontfeatures[\rmfamily]{Ligatures=TeX,Scale=1}
\fi
\usepackage{lmodern}
\ifPDFTeX\else
  % xetex/luatex font selection
\fi
% Use upquote if available, for straight quotes in verbatim environments
\IfFileExists{upquote.sty}{\usepackage{upquote}}{}
\IfFileExists{microtype.sty}{% use microtype if available
  \usepackage[]{microtype}
  \UseMicrotypeSet[protrusion]{basicmath} % disable protrusion for tt fonts
}{}
\makeatletter
\@ifundefined{KOMAClassName}{% if non-KOMA class
  \IfFileExists{parskip.sty}{%
    \usepackage{parskip}
  }{% else
    \setlength{\parindent}{0pt}
    \setlength{\parskip}{6pt plus 2pt minus 1pt}}
}{% if KOMA class
  \KOMAoptions{parskip=half}}
\makeatother
\usepackage{xcolor}
\usepackage[margin=1in]{geometry}
\usepackage{color}
\usepackage{fancyvrb}
\newcommand{\VerbBar}{|}
\newcommand{\VERB}{\Verb[commandchars=\\\{\}]}
\DefineVerbatimEnvironment{Highlighting}{Verbatim}{commandchars=\\\{\}}
% Add ',fontsize=\small' for more characters per line
\usepackage{framed}
\definecolor{shadecolor}{RGB}{248,248,248}
\newenvironment{Shaded}{\begin{snugshade}}{\end{snugshade}}
\newcommand{\AlertTok}[1]{\textcolor[rgb]{0.94,0.16,0.16}{#1}}
\newcommand{\AnnotationTok}[1]{\textcolor[rgb]{0.56,0.35,0.01}{\textbf{\textit{#1}}}}
\newcommand{\AttributeTok}[1]{\textcolor[rgb]{0.13,0.29,0.53}{#1}}
\newcommand{\BaseNTok}[1]{\textcolor[rgb]{0.00,0.00,0.81}{#1}}
\newcommand{\BuiltInTok}[1]{#1}
\newcommand{\CharTok}[1]{\textcolor[rgb]{0.31,0.60,0.02}{#1}}
\newcommand{\CommentTok}[1]{\textcolor[rgb]{0.56,0.35,0.01}{\textit{#1}}}
\newcommand{\CommentVarTok}[1]{\textcolor[rgb]{0.56,0.35,0.01}{\textbf{\textit{#1}}}}
\newcommand{\ConstantTok}[1]{\textcolor[rgb]{0.56,0.35,0.01}{#1}}
\newcommand{\ControlFlowTok}[1]{\textcolor[rgb]{0.13,0.29,0.53}{\textbf{#1}}}
\newcommand{\DataTypeTok}[1]{\textcolor[rgb]{0.13,0.29,0.53}{#1}}
\newcommand{\DecValTok}[1]{\textcolor[rgb]{0.00,0.00,0.81}{#1}}
\newcommand{\DocumentationTok}[1]{\textcolor[rgb]{0.56,0.35,0.01}{\textbf{\textit{#1}}}}
\newcommand{\ErrorTok}[1]{\textcolor[rgb]{0.64,0.00,0.00}{\textbf{#1}}}
\newcommand{\ExtensionTok}[1]{#1}
\newcommand{\FloatTok}[1]{\textcolor[rgb]{0.00,0.00,0.81}{#1}}
\newcommand{\FunctionTok}[1]{\textcolor[rgb]{0.13,0.29,0.53}{\textbf{#1}}}
\newcommand{\ImportTok}[1]{#1}
\newcommand{\InformationTok}[1]{\textcolor[rgb]{0.56,0.35,0.01}{\textbf{\textit{#1}}}}
\newcommand{\KeywordTok}[1]{\textcolor[rgb]{0.13,0.29,0.53}{\textbf{#1}}}
\newcommand{\NormalTok}[1]{#1}
\newcommand{\OperatorTok}[1]{\textcolor[rgb]{0.81,0.36,0.00}{\textbf{#1}}}
\newcommand{\OtherTok}[1]{\textcolor[rgb]{0.56,0.35,0.01}{#1}}
\newcommand{\PreprocessorTok}[1]{\textcolor[rgb]{0.56,0.35,0.01}{\textit{#1}}}
\newcommand{\RegionMarkerTok}[1]{#1}
\newcommand{\SpecialCharTok}[1]{\textcolor[rgb]{0.81,0.36,0.00}{\textbf{#1}}}
\newcommand{\SpecialStringTok}[1]{\textcolor[rgb]{0.31,0.60,0.02}{#1}}
\newcommand{\StringTok}[1]{\textcolor[rgb]{0.31,0.60,0.02}{#1}}
\newcommand{\VariableTok}[1]{\textcolor[rgb]{0.00,0.00,0.00}{#1}}
\newcommand{\VerbatimStringTok}[1]{\textcolor[rgb]{0.31,0.60,0.02}{#1}}
\newcommand{\WarningTok}[1]{\textcolor[rgb]{0.56,0.35,0.01}{\textbf{\textit{#1}}}}
\usepackage{longtable,booktabs,array}
\usepackage{calc} % for calculating minipage widths
% Correct order of tables after \paragraph or \subparagraph
\usepackage{etoolbox}
\makeatletter
\patchcmd\longtable{\par}{\if@noskipsec\mbox{}\fi\par}{}{}
\makeatother
% Allow footnotes in longtable head/foot
\IfFileExists{footnotehyper.sty}{\usepackage{footnotehyper}}{\usepackage{footnote}}
\makesavenoteenv{longtable}
\usepackage{graphicx}
\makeatletter
\def\maxwidth{\ifdim\Gin@nat@width>\linewidth\linewidth\else\Gin@nat@width\fi}
\def\maxheight{\ifdim\Gin@nat@height>\textheight\textheight\else\Gin@nat@height\fi}
\makeatother
% Scale images if necessary, so that they will not overflow the page
% margins by default, and it is still possible to overwrite the defaults
% using explicit options in \includegraphics[width, height, ...]{}
\setkeys{Gin}{width=\maxwidth,height=\maxheight,keepaspectratio}
% Set default figure placement to htbp
\makeatletter
\def\fps@figure{htbp}
\makeatother
\setlength{\emergencystretch}{3em} % prevent overfull lines
\providecommand{\tightlist}{%
  \setlength{\itemsep}{0pt}\setlength{\parskip}{0pt}}
\setcounter{secnumdepth}{-\maxdimen} % remove section numbering
\usepackage{ifthen}
\let\oldincludegraphics\includegraphics
\renewcommand{\includegraphics}[2][]{ \ifthenelse{ \equal{#1}{} } { \oldincludegraphics[width=2.5cm,height=2.5cm,keepaspectratio=true]{#2} } { \oldincludegraphics[#1]{#2} } }
\ifLuaTeX
  \usepackage{selnolig}  % disable illegal ligatures
\fi
\usepackage{bookmark}
\IfFileExists{xurl.sty}{\usepackage{xurl}}{} % add URL line breaks if available
\urlstyle{same}
\hypersetup{
  pdftitle={Tamanho mínimo de amostra},
  pdfauthor={Arquimedes Macedo. Tiago Rodrigues},
  hidelinks,
  pdfcreator={LaTeX via pandoc}}

\title{Tamanho mínimo de amostra}
\author{Arquimedes Macedo. Tiago Rodrigues}
\date{}

\begin{document}
\maketitle

{
\setcounter{tocdepth}{2}
\tableofcontents
}
\begin{Shaded}
\begin{Highlighting}[]
\FunctionTok{library}\NormalTok{(dplyr)}
\FunctionTok{library}\NormalTok{(tidyr)}
\FunctionTok{library}\NormalTok{(readxl)}
\FunctionTok{library}\NormalTok{(knitr)}
\FunctionTok{library}\NormalTok{(ggplot2)}
\FunctionTok{library}\NormalTok{(ggridges)}
\FunctionTok{library}\NormalTok{(reshape2)}
\FunctionTok{library}\NormalTok{(gridExtra)}
\FunctionTok{library}\NormalTok{(vtable)}
\FunctionTok{library}\NormalTok{(purrr)}

\CommentTok{\# Centering figures chunk output}
\NormalTok{knitr}\SpecialCharTok{::}\NormalTok{opts\_chunk}\SpecialCharTok{$}\FunctionTok{set}\NormalTok{(}\AttributeTok{out.height =} \StringTok{"}\SpecialCharTok{\textbackslash{}\textbackslash{}}\StringTok{textheight"}\NormalTok{,  }\AttributeTok{out.width =} \StringTok{"}\SpecialCharTok{\textbackslash{}\textbackslash{}}\StringTok{textwidth"}\NormalTok{,}
                      \AttributeTok{out.extra =} \StringTok{"keepaspectratio=true"}\NormalTok{, }\AttributeTok{fig.align =} \StringTok{"center"}\NormalTok{)}
\end{Highlighting}
\end{Shaded}

\begin{Shaded}
\begin{Highlighting}[]
\NormalTok{theme.base }\OtherTok{\textless{}{-}} \FunctionTok{theme\_minimal}\NormalTok{(}\AttributeTok{base\_size =} \DecValTok{11}\NormalTok{) }\SpecialCharTok{+}
  \FunctionTok{theme}\NormalTok{(}
    \AttributeTok{axis.text =} \FunctionTok{element\_text}\NormalTok{(}\AttributeTok{size =} \DecValTok{8}\NormalTok{),}
    \AttributeTok{plot.title =} \FunctionTok{element\_text}\NormalTok{(}\AttributeTok{hjust =} \FloatTok{0.5}\NormalTok{, }\AttributeTok{size =} \DecValTok{10}\NormalTok{, }\AttributeTok{face =} \StringTok{"bold"}\NormalTok{),}
    \AttributeTok{axis.title =} \FunctionTok{element\_text}\NormalTok{(}\AttributeTok{size =} \DecValTok{10}\NormalTok{),}
    \AttributeTok{panel.grid.major =} \FunctionTok{element\_line}\NormalTok{(}\AttributeTok{colour =} \StringTok{"grey90"}\NormalTok{, }\AttributeTok{linewidth =} \FloatTok{0.5}\NormalTok{),}
    \AttributeTok{panel.grid.minor =} \FunctionTok{element\_line}\NormalTok{(}\AttributeTok{colour =} \FunctionTok{adjustcolor}\NormalTok{(}\StringTok{"grey90"}\NormalTok{, }\AttributeTok{alpha.f =} \FloatTok{0.5}\NormalTok{), }\AttributeTok{linewidth =} \FloatTok{0.25}\NormalTok{),}
    \AttributeTok{panel.border =} \FunctionTok{element\_blank}\NormalTok{(),}
    \AttributeTok{panel.background =} \FunctionTok{element\_blank}\NormalTok{(),}
    \AttributeTok{plot.background =} \FunctionTok{element\_blank}\NormalTok{(),}
    \AttributeTok{axis.line.x =} \FunctionTok{element\_line}\NormalTok{(}\AttributeTok{colour =} \StringTok{"grey"}\NormalTok{),}
    \AttributeTok{axis.line.y =} \FunctionTok{element\_line}\NormalTok{(}\AttributeTok{colour =} \StringTok{"grey"}\NormalTok{),}
\NormalTok{  )}

\NormalTok{theme.no\_legend }\OtherTok{\textless{}{-}} \FunctionTok{theme}\NormalTok{(}\AttributeTok{legend.position =} \StringTok{"none"}\NormalTok{)}

\NormalTok{theme.no\_grid }\OtherTok{\textless{}{-}}  \FunctionTok{theme}\NormalTok{(}
  \AttributeTok{panel.grid.major =} \FunctionTok{element\_blank}\NormalTok{(),}
  \AttributeTok{panel.grid.minor =} \FunctionTok{element\_blank}\NormalTok{()}
\NormalTok{)}

\NormalTok{theme.no\_axis }\OtherTok{\textless{}{-}} \FunctionTok{theme}\NormalTok{(}
  \AttributeTok{axis.line.x =} \FunctionTok{element\_blank}\NormalTok{(),}
  \AttributeTok{axis.line.y =} \FunctionTok{element\_blank}\NormalTok{()}
\NormalTok{)}

\CommentTok{\# Theme for timeseries with legend}
\NormalTok{apply.theme.ts.legend }\OtherTok{\textless{}{-}} \ControlFlowTok{function}\NormalTok{() \{}
  \FunctionTok{list}\NormalTok{(}
    \FunctionTok{scale\_x\_date}\NormalTok{(}\AttributeTok{date\_labels =} \StringTok{"\%b \%d"}\NormalTok{, }\AttributeTok{date\_breaks =} \StringTok{"1 week"}\NormalTok{),}
\NormalTok{    theme.base }\SpecialCharTok{+}
      \FunctionTok{theme}\NormalTok{(}
        \AttributeTok{axis.text.x =} \FunctionTok{element\_text}\NormalTok{(}\AttributeTok{angle =} \DecValTok{45}\NormalTok{, }\AttributeTok{hjust =} \DecValTok{1}\NormalTok{),}
        \AttributeTok{panel.grid.major.x =} \FunctionTok{element\_blank}\NormalTok{(),}
        \AttributeTok{panel.grid.minor.x =} \FunctionTok{element\_blank}\NormalTok{()}
\NormalTok{      )}
\NormalTok{  )}
\NormalTok{\}}


\CommentTok{\# Theme for timeseries}
\NormalTok{apply.theme.ts }\OtherTok{\textless{}{-}} \ControlFlowTok{function}\NormalTok{() \{}
  \FunctionTok{list}\NormalTok{(}
    \FunctionTok{apply.theme.ts.legend}\NormalTok{(),}
\NormalTok{    theme.no\_legend}
\NormalTok{  )}
\NormalTok{\}}
\end{Highlighting}
\end{Shaded}

\subsection{Objetivo}\label{objetivo}

Estimar a quantidade média de leads diários, com 80\% de confiança, por
anunciante, de anúncios de vendas de imóveis na cidade de Florianópolis
(SC).

Com um erro máximo de 0.05, usando Amostragem Aleatório Simples sem
Reposição (AASs).

\emph{Lead}: é um contato de um cliente em potencial que demonstrou
interesse em um produto ou serviço.

IC de 80\% foi escolhido devido à falta de informações (descrita logo
mais), e também por ser este o valor máximo recomendado pela ABNT para
avaliações de imóveis (NBR 14653).

\subsection{Metodologia}\label{metodologia}

Almeja-se, a partir de uma lista de anúncios, realizar uma busca diária
de leads, usando uma amostra dos anúncios, e, a partir destes dados,
estimar a quantidade média de leads.

No entanto, entende-se que há limitações nas informações disponíveis,
como:

\begin{itemize}
\tightlist
\item
  O número de leads por anúncio.
\item
  Tempo total que o anúncio ficou ativo.
\item
  A sazonalidade do mercado (oferta e demanda).
\item
  A eficácia do anúncio (qualidade do anúncio, preço, localização, etc).
\item
  A qualidade dos leads (interesse real ou apenas curiosidade).
\item
  A distribuição subjacente dos leads ao longo do tempo.
\end{itemize}

Desta forma como um estudo piloto, foram obtidos leads diários, entre
Janeiro e Julho de 2024, de anúncios de um anunciante na cidade alvo.

\subsubsection{Análise exploratória}\label{anuxe1lise-exploratuxf3ria}

O banco de dados é composto por 3 colunas:

\begin{itemize}
\tightlist
\item
  \texttt{id\_registro}: identificador do lead.
\item
  \texttt{data\_criado\_em}: dia que o lead foi gerado.
\item
  \texttt{id\_anuncio}: identificador do anúncio.
\end{itemize}

\paragraph{Amostra dos dados}\label{amostra-dos-dados}

\begin{Shaded}
\begin{Highlighting}[]
\NormalTok{df\_leads }\OtherTok{\textless{}{-}} \FunctionTok{read\_excel}\NormalTok{(}\StringTok{"dataset/leads.xlsx"}\NormalTok{, }\AttributeTok{col\_types =} \FunctionTok{c}\NormalTok{(}\StringTok{"numeric"}\NormalTok{, }\StringTok{"date"}\NormalTok{, }\StringTok{"text"}\NormalTok{))}
\NormalTok{df\_leads}\SpecialCharTok{$}\NormalTok{data\_criado\_em }\OtherTok{\textless{}{-}} \FunctionTok{as.Date}\NormalTok{(df\_leads}\SpecialCharTok{$}\NormalTok{data\_criado\_em)}
\FunctionTok{kable}\NormalTok{(}\FunctionTok{head}\NormalTok{(df\_leads))}
\end{Highlighting}
\end{Shaded}

\begin{longtable}[]{@{}rll@{}}
\toprule\noalign{}
id\_registro & data\_criado\_em & id\_anuncio \\
\midrule\noalign{}
\endhead
\bottomrule\noalign{}
\endlastfoot
1 & 2024-01-18 & LRB3GK \\
2 & 2024-01-19 & 4I931S \\
3 & 2024-01-19 & 4WUWGH \\
4 & 2024-01-19 & XNI94R \\
5 & 2024-01-20 & HRDJQG \\
6 & 2024-01-20 & CH8NIW \\
\end{longtable}

\paragraph{Leads diários}\label{leads-diuxe1rios}

\begin{Shaded}
\begin{Highlighting}[]
\NormalTok{df\_leads }\SpecialCharTok{\%\textgreater{}\%}
  \FunctionTok{group\_by}\NormalTok{(data\_criado\_em) }\SpecialCharTok{\%\textgreater{}\%}
  \FunctionTok{summarise}\NormalTok{(}\AttributeTok{leads =} \FunctionTok{n}\NormalTok{()) }\SpecialCharTok{\%\textgreater{}\%}
  \FunctionTok{ggplot}\NormalTok{(}\FunctionTok{aes}\NormalTok{(data\_criado\_em, leads)) }\SpecialCharTok{+}
  \FunctionTok{geom\_line}\NormalTok{(}\AttributeTok{color =} \StringTok{"royalblue"}\NormalTok{, }\AttributeTok{linewidth =} \FloatTok{0.5}\NormalTok{) }\SpecialCharTok{+}
  \FunctionTok{labs}\NormalTok{(}\AttributeTok{title =} \StringTok{"Leads diários"}\NormalTok{,}
       \AttributeTok{x =} \StringTok{"Dia"}\NormalTok{,}
       \AttributeTok{y =} \StringTok{"Leads"}\NormalTok{) }\SpecialCharTok{+}
  \FunctionTok{apply.theme.ts}\NormalTok{()}
\end{Highlighting}
\end{Shaded}

\begin{center} \ifthenelse{ \equal{width=\textwidth,height=\textheight,keepaspectratio=true}{} } { \includegraphics[width=2.5cm,height=2.5cm,keepaspectratio=true]{tamanho-amostra_files/figure-latex/unnamed-chunk-4-1} } { \includegraphics[width=\textwidth,height=\textheight,keepaspectratio=true]{tamanho-amostra_files/figure-latex/unnamed-chunk-4-1} }  \end{center}

Notam-se picos em intervalos semi-regulares, o que pode indicar
sazonalidade ou eventos específicos. Além disso, em Julho, houve uma
alta variabilidade nos leads diários.

\paragraph{Leads por anúncio}\label{leads-por-anuxfancio}

Vamos analisar a média diária de leads por anúncio.

Para isso, dividimos o número total de leads pelo número de anúncios
únicos para cada dia.

\begin{Shaded}
\begin{Highlighting}[]
\NormalTok{df\_leads\_incorrect\_mean }\OtherTok{\textless{}{-}}\NormalTok{ df\_leads }\SpecialCharTok{\%\textgreater{}\%}
  \FunctionTok{group\_by}\NormalTok{(data\_criado\_em) }\SpecialCharTok{\%\textgreater{}\%}
  \FunctionTok{summarise}\NormalTok{(}\AttributeTok{mean =} \FunctionTok{n}\NormalTok{()}\SpecialCharTok{/}\FunctionTok{length}\NormalTok{(}\FunctionTok{unique}\NormalTok{(id\_anuncio)))}

\FunctionTok{grid.arrange}\NormalTok{(}
\NormalTok{  df\_leads\_incorrect\_mean }\SpecialCharTok{\%\textgreater{}\%}
    \FunctionTok{ggplot}\NormalTok{(}\FunctionTok{aes}\NormalTok{(data\_criado\_em, mean)) }\SpecialCharTok{+}
    \FunctionTok{geom\_line}\NormalTok{(}\AttributeTok{color =} \StringTok{"royalblue"}\NormalTok{, }\AttributeTok{linewidth =} \FloatTok{0.5}\NormalTok{) }\SpecialCharTok{+}
    \FunctionTok{labs}\NormalTok{(}\AttributeTok{title =} \StringTok{"Média de leads por dia"}\NormalTok{,}
         \AttributeTok{x =} \StringTok{"Dia"}\NormalTok{,}
         \AttributeTok{y =} \StringTok{"Média de leads"}\NormalTok{) }\SpecialCharTok{+}
    \FunctionTok{apply.theme.ts}\NormalTok{(),}
\NormalTok{  df\_leads\_incorrect\_mean }\SpecialCharTok{\%\textgreater{}\%}
    \FunctionTok{ggplot}\NormalTok{(}\FunctionTok{aes}\NormalTok{(mean)) }\SpecialCharTok{+}
    \FunctionTok{geom\_histogram}\NormalTok{(}\AttributeTok{bins =} \DecValTok{30}\NormalTok{, }\AttributeTok{color =} \StringTok{"royalblue"}\NormalTok{, }\AttributeTok{fill =} \StringTok{"royalblue"}\NormalTok{, }\AttributeTok{alpha =} \FloatTok{0.5}\NormalTok{) }\SpecialCharTok{+}
    \FunctionTok{labs}\NormalTok{(}\AttributeTok{title =} \StringTok{""}\NormalTok{,}
         \AttributeTok{x =} \StringTok{"Média de leads"}\NormalTok{,}
         \AttributeTok{y =} \StringTok{""}\NormalTok{) }\SpecialCharTok{+}
\NormalTok{    theme.base }\SpecialCharTok{+}\NormalTok{ theme.no\_legend,}
  \AttributeTok{nrow =} \DecValTok{2}
\NormalTok{)}
\end{Highlighting}
\end{Shaded}

\begin{center} \ifthenelse{ \equal{width=\textwidth,height=\textheight,keepaspectratio=true}{} } { \includegraphics[width=2.5cm,height=2.5cm,keepaspectratio=true]{tamanho-amostra_files/figure-latex/unnamed-chunk-5-1} } { \includegraphics[width=\textwidth,height=\textheight,keepaspectratio=true]{tamanho-amostra_files/figure-latex/unnamed-chunk-5-1} }  \end{center}

Será que é só isso mesmo?

\includegraphics{./images/flork-pensando.jpg}

Claro que não! A média diária de leads por anúncio é uma estimativa
incorreta, pois não considera a quantidade de anúncios ativos em cada
dia, e acaba gerando um viés.

\subsubsection{Estimativa da média}\label{estimativa-da-muxe9dia}

Para corrigir o problema anterior, vamos completar os dados com zeros
para os dias sem leads.

Isto é, vamos pegar o primeiro e o último dia que o anúncio teve leads,
e criar novos registros entre estas datas, para dias sem lead.

\begin{Shaded}
\begin{Highlighting}[]
\NormalTok{df\_leads\_complete }\OtherTok{\textless{}{-}}\NormalTok{ df\_leads }\SpecialCharTok{\%\textgreater{}\%}
  \FunctionTok{group\_by}\NormalTok{(id\_anuncio, data\_criado\_em) }\SpecialCharTok{\%\textgreater{}\%}
  \FunctionTok{summarise}\NormalTok{(}\AttributeTok{leads =} \FunctionTok{n}\NormalTok{(), }\AttributeTok{.groups =} \StringTok{\textquotesingle{}drop\textquotesingle{}}\NormalTok{) }\SpecialCharTok{\%\textgreater{}\%}
  \FunctionTok{group\_by}\NormalTok{(id\_anuncio) }\SpecialCharTok{\%\textgreater{}\%}
  \CommentTok{\# Creates a list of dataframes by id}
\NormalTok{  tidyr}\SpecialCharTok{::}\FunctionTok{nest}\NormalTok{() }\SpecialCharTok{\%\textgreater{}\%}
  \FunctionTok{mutate}\NormalTok{(}
    \CommentTok{\# Creates a sequence of dates by id}
    \AttributeTok{date\_seq =} \FunctionTok{map}\NormalTok{(data, }\SpecialCharTok{\textasciitilde{}}\FunctionTok{seq}\NormalTok{(}\FunctionTok{min}\NormalTok{(.}\SpecialCharTok{$}\NormalTok{data\_criado\_em), }\FunctionTok{max}\NormalTok{(.}\SpecialCharTok{$}\NormalTok{data\_criado\_em), }\AttributeTok{by =} \StringTok{"day"}\NormalTok{)),}
    \CommentTok{\# Completes the missing dates}
    \AttributeTok{data =} \FunctionTok{map2}\NormalTok{(}
\NormalTok{      data, date\_seq,}
\NormalTok{      \textbackslash{}(data\_, seq\_) \{}
\NormalTok{        data\_ }\SpecialCharTok{\%\textgreater{}\%}
          \FunctionTok{complete}\NormalTok{(}\AttributeTok{data\_criado\_em =}\NormalTok{ seq\_, }\AttributeTok{fill =} \FunctionTok{list}\NormalTok{(}\AttributeTok{leads =} \DecValTok{0}\NormalTok{))}
\NormalTok{      \}}
\NormalTok{    )}
\NormalTok{  ) }\SpecialCharTok{\%\textgreater{}\%}
  \CommentTok{\# Removes the auxiliary column}
  \FunctionTok{select}\NormalTok{(}\SpecialCharTok{{-}}\NormalTok{date\_seq) }\SpecialCharTok{\%\textgreater{}\%}
  \CommentTok{\# Unnests the data}
  \FunctionTok{unnest}\NormalTok{(data)}

\FunctionTok{kable}\NormalTok{(}\FunctionTok{head}\NormalTok{(df\_leads\_complete))}
\end{Highlighting}
\end{Shaded}

\begin{longtable}[]{@{}llr@{}}
\toprule\noalign{}
id\_anuncio & data\_criado\_em & leads \\
\midrule\noalign{}
\endhead
\bottomrule\noalign{}
\endlastfoot
00OPP2 & 2024-03-10 & 1 \\
00SLR7 & 2024-06-29 & 1 \\
00TPRF & 2024-06-23 & 1 \\
02NTL4 & 2024-04-20 & 1 \\
02NTL4 & 2024-04-21 & 0 \\
02NTL4 & 2024-04-22 & 0 \\
\end{longtable}

A partir desta correção, temos as seguintes médias diárias.

\begin{Shaded}
\begin{Highlighting}[]
\NormalTok{df\_leads\_daily }\OtherTok{\textless{}{-}}\NormalTok{ df\_leads\_complete }\SpecialCharTok{\%\textgreater{}\%}
  \FunctionTok{group\_by}\NormalTok{(data\_criado\_em) }\SpecialCharTok{\%\textgreater{}\%}
  \FunctionTok{summarise}\NormalTok{(}\AttributeTok{mean =} \FunctionTok{mean}\NormalTok{(leads),}
            \AttributeTok{total\_leads =} \FunctionTok{sum}\NormalTok{(leads),}
            \AttributeTok{active\_listings =} \FunctionTok{n\_distinct}\NormalTok{(id\_anuncio))}

\FunctionTok{grid.arrange}\NormalTok{(}
\NormalTok{  df\_leads\_daily }\SpecialCharTok{\%\textgreater{}\%}
    \FunctionTok{ggplot}\NormalTok{(}\FunctionTok{aes}\NormalTok{(data\_criado\_em, mean)) }\SpecialCharTok{+}
    \FunctionTok{geom\_line}\NormalTok{(}\AttributeTok{color =} \StringTok{"royalblue"}\NormalTok{, }\AttributeTok{linewidth =} \FloatTok{0.5}\NormalTok{) }\SpecialCharTok{+}
    \FunctionTok{labs}\NormalTok{(}\AttributeTok{title =} \StringTok{"Média de leads por dia"}\NormalTok{,}
         \AttributeTok{x =} \StringTok{"Dia"}\NormalTok{,}
         \AttributeTok{y =} \StringTok{"Média de leads"}\NormalTok{) }\SpecialCharTok{+}
    \FunctionTok{apply.theme.ts}\NormalTok{(),}
\NormalTok{  df\_leads\_daily }\SpecialCharTok{\%\textgreater{}\%}
    \FunctionTok{ggplot}\NormalTok{(}\FunctionTok{aes}\NormalTok{(mean)) }\SpecialCharTok{+}
    \FunctionTok{geom\_histogram}\NormalTok{(}\AttributeTok{bins =} \DecValTok{20}\NormalTok{, }\AttributeTok{color =} \StringTok{"royalblue"}\NormalTok{, }\AttributeTok{fill =} \StringTok{"royalblue"}\NormalTok{, }\AttributeTok{alpha =} \FloatTok{0.5}\NormalTok{) }\SpecialCharTok{+}
    \FunctionTok{labs}\NormalTok{(}\AttributeTok{title =} \StringTok{""}\NormalTok{,}
         \AttributeTok{x =} \StringTok{"Média de leads"}\NormalTok{,}
         \AttributeTok{y =} \StringTok{""}\NormalTok{) }\SpecialCharTok{+}
\NormalTok{    theme.base }\SpecialCharTok{+}\NormalTok{ theme.no\_legend,}
  \AttributeTok{nrow =} \DecValTok{2}
\NormalTok{)}
\end{Highlighting}
\end{Shaded}

\begin{center} \ifthenelse{ \equal{width=\textwidth,height=\textheight,keepaspectratio=true}{} } { \includegraphics[width=2.5cm,height=2.5cm,keepaspectratio=true]{tamanho-amostra_files/figure-latex/unnamed-chunk-7-1} } { \includegraphics[width=\textwidth,height=\textheight,keepaspectratio=true]{tamanho-amostra_files/figure-latex/unnamed-chunk-7-1} }  \end{center}

Mas não está totalmente correto\ldots{}

\includegraphics{./images/flork-exercer-a-calma.jpg}

Lembrando que esta é uma aproximação e não corresponde totalmente ao que
de fato aconteceu, para computar a verdadeira média, precisariamos da
listagem de todos os anúncios ativos no dia.

Nota-se, também, que existem pontos extremos no início e no fim da
série, isso pode ser explicado por anúncios que estavam ativos antes do
início do período analisado ou que apareceram um pouco antes do fim.

Portanto vamos analisar apenas entre 01/02/2024 e 20/07/2024.

\begin{Shaded}
\begin{Highlighting}[]
\NormalTok{df\_leads\_complete\_filtered }\OtherTok{\textless{}{-}}\NormalTok{ df\_leads\_complete }\SpecialCharTok{\%\textgreater{}\%}
  \FunctionTok{filter}\NormalTok{(}\FunctionTok{between}\NormalTok{(data\_criado\_em, }\FunctionTok{as.Date}\NormalTok{(}\StringTok{"2024{-}02{-}01"}\NormalTok{), }\FunctionTok{as.Date}\NormalTok{(}\StringTok{"2024{-}07{-}20"}\NormalTok{)))}

\NormalTok{df\_leads\_daily\_filtered }\OtherTok{\textless{}{-}}\NormalTok{ df\_leads\_complete\_filtered }\SpecialCharTok{\%\textgreater{}\%}
  \FunctionTok{group\_by}\NormalTok{(data\_criado\_em) }\SpecialCharTok{\%\textgreater{}\%}
  \FunctionTok{summarise}\NormalTok{(}\AttributeTok{mean =} \FunctionTok{mean}\NormalTok{(leads),}
            \AttributeTok{total\_leads =} \FunctionTok{sum}\NormalTok{(leads),}
            \AttributeTok{active\_listings =} \FunctionTok{n\_distinct}\NormalTok{(id\_anuncio))}
\end{Highlighting}
\end{Shaded}

\begin{Shaded}
\begin{Highlighting}[]
\FunctionTok{sumtable}\NormalTok{(df\_leads\_complete\_filtered, }\AttributeTok{add.median =}\NormalTok{ T, }\AttributeTok{title =} \StringTok{"Registros corrigidos"}\NormalTok{)}
\end{Highlighting}
\end{Shaded}

\begin{table}

\caption{\label{tab:unnamed-chunk-9}Registros corrigidos}
\centering
\begin{tabular}[t]{lllllllll}
\toprule
Variable & N & Mean & Std. Dev. & Min & Pctl. 25 & Pctl. 50 & Pctl. 75 & Max\\
\midrule
leads & 11253 & 0.12 & 0.34 & 0 & 0 & 0 & 0 & 4\\
\bottomrule
\end{tabular}
\end{table}

\begin{Shaded}
\begin{Highlighting}[]
\FunctionTok{grid.arrange}\NormalTok{(}
\NormalTok{  df\_leads\_daily\_filtered }\SpecialCharTok{\%\textgreater{}\%}
    \FunctionTok{ggplot}\NormalTok{(}\FunctionTok{aes}\NormalTok{(data\_criado\_em, mean)) }\SpecialCharTok{+}
    \FunctionTok{geom\_line}\NormalTok{(}\AttributeTok{color =} \StringTok{"royalblue"}\NormalTok{, }\AttributeTok{linewidth =} \FloatTok{0.5}\NormalTok{) }\SpecialCharTok{+}
    \FunctionTok{labs}\NormalTok{(}\AttributeTok{title =} \StringTok{"Média de leads por dia"}\NormalTok{,}
         \AttributeTok{x =} \StringTok{"Dia"}\NormalTok{,}
         \AttributeTok{y =} \StringTok{"Média de leads"}\NormalTok{) }\SpecialCharTok{+}
    \FunctionTok{apply.theme.ts}\NormalTok{(),}
\NormalTok{  df\_leads\_daily\_filtered }\SpecialCharTok{\%\textgreater{}\%}
    \FunctionTok{ggplot}\NormalTok{(}\FunctionTok{aes}\NormalTok{(mean)) }\SpecialCharTok{+}
    \FunctionTok{coord\_cartesian}\NormalTok{(}\AttributeTok{xlim =} \FunctionTok{c}\NormalTok{(}\SpecialCharTok{{-}}\FloatTok{0.01}\NormalTok{, }\FloatTok{0.4}\NormalTok{)) }\SpecialCharTok{+}
    \FunctionTok{geom\_histogram}\NormalTok{(}\AttributeTok{bins =} \DecValTok{20}\NormalTok{, }\AttributeTok{color =} \FunctionTok{adjustcolor}\NormalTok{(}\StringTok{"royalblue"}\NormalTok{, }\AttributeTok{alpha.f =} \FloatTok{0.3}\NormalTok{), }\AttributeTok{fill =} \StringTok{"royalblue"}\NormalTok{, }\AttributeTok{alpha =} \FloatTok{0.5}\NormalTok{) }\SpecialCharTok{+}
    \FunctionTok{labs}\NormalTok{(}\AttributeTok{title =} \StringTok{""}\NormalTok{,}
         \AttributeTok{x =} \StringTok{""}\NormalTok{,}
         \AttributeTok{y =} \StringTok{""}\NormalTok{) }\SpecialCharTok{+}
\NormalTok{    theme.base }\SpecialCharTok{+}\NormalTok{ theme.no\_legend }\SpecialCharTok{+}\NormalTok{ theme.no\_axis }\SpecialCharTok{+}
    \FunctionTok{theme}\NormalTok{(}\AttributeTok{panel.grid.minor.y =} \FunctionTok{element\_blank}\NormalTok{()),}
\NormalTok{  df\_leads\_daily\_filtered }\SpecialCharTok{\%\textgreater{}\%}
    \FunctionTok{ggplot}\NormalTok{(}\FunctionTok{aes}\NormalTok{(mean)) }\SpecialCharTok{+}
    \FunctionTok{coord\_cartesian}\NormalTok{(}\AttributeTok{xlim =} \FunctionTok{c}\NormalTok{(}\SpecialCharTok{{-}}\FloatTok{0.02}\NormalTok{, }\FloatTok{0.4}\NormalTok{)) }\SpecialCharTok{+}
    \FunctionTok{geom\_boxplot}\NormalTok{(}\AttributeTok{color =} \FunctionTok{adjustcolor}\NormalTok{(}\StringTok{"royalblue"}\NormalTok{, }\AttributeTok{alpha.f =} \FloatTok{0.8}\NormalTok{), }\AttributeTok{fill =} \StringTok{"royalblue"}\NormalTok{, }\AttributeTok{alpha =} \FloatTok{0.5}\NormalTok{) }\SpecialCharTok{+}
    \FunctionTok{labs}\NormalTok{(}\AttributeTok{title =} \StringTok{""}\NormalTok{,}
         \AttributeTok{x =} \StringTok{""}\NormalTok{,}
         \AttributeTok{y =} \StringTok{""}\NormalTok{) }\SpecialCharTok{+}
\NormalTok{    theme.base }\SpecialCharTok{+}\NormalTok{ theme.no\_legend }\SpecialCharTok{+}\NormalTok{ theme.no\_axis }\SpecialCharTok{+}
    \FunctionTok{theme}\NormalTok{(}\AttributeTok{axis.text.y =} \FunctionTok{element\_blank}\NormalTok{(),}
          \AttributeTok{axis.ticks.y =} \FunctionTok{element\_blank}\NormalTok{(),}
          \AttributeTok{panel.grid.major.y =} \FunctionTok{element\_blank}\NormalTok{(),}
          \AttributeTok{panel.grid.minor.y =} \FunctionTok{element\_blank}\NormalTok{()),}
  \AttributeTok{nrow =} \DecValTok{3}\NormalTok{,}
  \AttributeTok{heights =} \FunctionTok{c}\NormalTok{(}\DecValTok{3}\NormalTok{, }\DecValTok{2}\NormalTok{, }\FloatTok{1.5}\NormalTok{)}
\NormalTok{)}
\end{Highlighting}
\end{Shaded}

\begin{center} \ifthenelse{ \equal{width=\textwidth,height=\textheight,keepaspectratio=true}{} } { \includegraphics[width=2.5cm,height=2.5cm,keepaspectratio=true]{tamanho-amostra_files/figure-latex/unnamed-chunk-10-1} } { \includegraphics[width=\textwidth,height=\textheight,keepaspectratio=true]{tamanho-amostra_files/figure-latex/unnamed-chunk-10-1} }  \end{center}

\begin{Shaded}
\begin{Highlighting}[]
\NormalTok{leads\_daily\_mean }\OtherTok{\textless{}{-}} \FunctionTok{mean}\NormalTok{(df\_leads\_complete\_filtered}\SpecialCharTok{$}\NormalTok{leads)}
\NormalTok{leads\_daily\_sd }\OtherTok{\textless{}{-}} \FunctionTok{sd}\NormalTok{(df\_leads\_complete\_filtered}\SpecialCharTok{$}\NormalTok{leads)}

\NormalTok{leads\_mean\_of\_means }\OtherTok{\textless{}{-}} \FunctionTok{mean}\NormalTok{(df\_leads\_daily\_filtered}\SpecialCharTok{$}\NormalTok{mean)}
\NormalTok{leads\_sd\_of\_means }\OtherTok{\textless{}{-}} \FunctionTok{sd}\NormalTok{(df\_leads\_daily\_filtered}\SpecialCharTok{$}\NormalTok{mean)}
\NormalTok{mean\_active\_listings }\OtherTok{\textless{}{-}} \FunctionTok{ceiling}\NormalTok{(}\FunctionTok{mean}\NormalTok{(df\_leads\_daily\_filtered}\SpecialCharTok{$}\NormalTok{active\_listings))}

\FunctionTok{sumtable}\NormalTok{(df\_leads\_daily\_filtered, }\AttributeTok{add.median =}\NormalTok{ T, }\AttributeTok{title =} \StringTok{"Média de leads por dia"}\NormalTok{)}
\end{Highlighting}
\end{Shaded}

\begin{table}

\caption{\label{tab:unnamed-chunk-11}Média de leads por dia}
\centering
\begin{tabular}[t]{lllllllll}
\toprule
Variable & N & Mean & Std. Dev. & Min & Pctl. 25 & Pctl. 50 & Pctl. 75 & Max\\
\midrule
mean & 171 & 0.12 & 0.069 & 0 & 0.068 & 0.11 & 0.16 & 0.35\\
total\_leads & 171 & 7.8 & 5.1 & 0 & 4 & 7 & 10 & 28\\
active\_listings & 171 & 66 & 14 & 25 & 58 & 71 & 76 & 93\\
\bottomrule
\end{tabular}
\end{table}

\subsection{Resultados}\label{resultados}

Assim, apesar dos pesares, temos uma média de
\texttt{\textasciitilde{}0.118} leads por dia, com um desvio padrão de
\texttt{\textasciitilde{}0.337}. Além disso, a média das médias diárias
é de \texttt{\textasciitilde{}0.118} com um desvio padrão de
\texttt{\textasciitilde{}0.069}.

\subsubsection{Tamanho da amostra}\label{tamanho-da-amostra}

Calculamos que o tamanho da amostra, a partir da equação

\[
\begin{aligned}
n' &\ge \left(Z_{\alpha / 2} \frac{\sigma}{\text{e}}\right)^2 \\
n  &= n' \cdot \frac{N-n}{N-1}
\end{aligned}
\]

onde, \(Z_{\alpha / 2}\) é o valor crítico da distribuição normal,
\(\text{e}\) é a margem de erro, \(\sigma\) é o desvio padrão, \(N\) é o
tamanho da população, \(n'\) é o tamanho da amostra com amostra
aleatória simples com reposição (AASc), e, \(n\) é o tamanho da amostra
sem reposição (AASs).

\begin{Shaded}
\begin{Highlighting}[]
\NormalTok{confidence }\OtherTok{\textless{}{-}} \FloatTok{0.8}
\NormalTok{z\_quartil }\OtherTok{\textless{}{-}} \FunctionTok{abs}\NormalTok{(}\FunctionTok{qnorm}\NormalTok{((}\DecValTok{1} \SpecialCharTok{{-}}\NormalTok{ confidence) }\SpecialCharTok{/} \DecValTok{2}\NormalTok{))}
\NormalTok{max\_error }\OtherTok{\textless{}{-}} \FloatTok{0.05}
\NormalTok{sigma }\OtherTok{\textless{}{-}}\NormalTok{ leads\_sd\_of\_means}
\NormalTok{size\_population }\OtherTok{\textless{}{-}}\NormalTok{ mean\_active\_listings}

\NormalTok{computed\_sample\_size\_aasc }\OtherTok{\textless{}{-}}\NormalTok{ (z\_quartil }\SpecialCharTok{*}\NormalTok{ sigma }\SpecialCharTok{/}\NormalTok{ max\_error)}\SpecialCharTok{\^{}}\DecValTok{2}
\NormalTok{computed\_sample\_size }\OtherTok{\textless{}{-}} \FunctionTok{ceiling}\NormalTok{(}
\NormalTok{  computed\_sample\_size\_aasc }\SpecialCharTok{*}
\NormalTok{    (size\_population }\SpecialCharTok{{-}}\NormalTok{ computed\_sample\_size\_aasc) }\SpecialCharTok{/}\NormalTok{ (size\_population }\SpecialCharTok{{-}} \DecValTok{1}\NormalTok{)}
\NormalTok{)}
\end{Highlighting}
\end{Shaded}

Lembrando que queremos estimar a média de leads diários, portanto, vamos
usar a média das médias diárias.

\textbf{Ta-dá!} Para obter uma margem de erro de \texttt{0.05} com
\texttt{80\%} de confiança, utilizando AASs, precisamos de uma amostra
de \texttt{4} anúncios.

\includegraphics{./images/flork-orgulhoso.png}

\subsubsection{Análise do erro}\label{anuxe1lise-do-erro}

Vamos analisar o erro da média de leads diários, a partir do tamanho da
amostra calculado.

\begin{Shaded}
\begin{Highlighting}[]
\NormalTok{sample\_repetitions }\OtherTok{\textless{}{-}} \DecValTok{20}

\NormalTok{df\_leads\_daily\_error }\OtherTok{\textless{}{-}}\NormalTok{ df\_leads\_complete\_filtered }\SpecialCharTok{\%\textgreater{}\%}
  \FunctionTok{group\_by}\NormalTok{(data\_criado\_em) }\SpecialCharTok{\%\textgreater{}\%}
  \FunctionTok{summarise}\NormalTok{(}\AttributeTok{lead\_mean =} \FunctionTok{mean}\NormalTok{(leads),}
            \AttributeTok{leads =} \FunctionTok{list}\NormalTok{(leads),}
            \AttributeTok{active\_listings =} \FunctionTok{n\_distinct}\NormalTok{(id\_anuncio)) }\SpecialCharTok{\%\textgreater{}\%}
  \FunctionTok{rowwise}\NormalTok{() }\SpecialCharTok{\%\textgreater{}\%}
  \FunctionTok{mutate}\NormalTok{(}
    \AttributeTok{samples =} \FunctionTok{list}\NormalTok{(}\FunctionTok{replicate}\NormalTok{(sample\_repetitions, }\FunctionTok{sample}\NormalTok{(leads, computed\_sample\_size, }\AttributeTok{replace =}\NormalTok{ F), }\AttributeTok{simplify =}\NormalTok{ F)),}
    \AttributeTok{samples\_mean =} \FunctionTok{list}\NormalTok{(}\FunctionTok{map\_dbl}\NormalTok{(samples, mean)),}
    \AttributeTok{samples\_error =} \FunctionTok{list}\NormalTok{(}\FunctionTok{map\_dbl}\NormalTok{(samples\_mean, }\SpecialCharTok{\textasciitilde{}} \FunctionTok{abs}\NormalTok{(.x }\SpecialCharTok{{-}}\NormalTok{ lead\_mean))),}
    \AttributeTok{min\_mean =} \FunctionTok{min}\NormalTok{(samples\_mean),}
    \AttributeTok{max\_mean =} \FunctionTok{max}\NormalTok{(samples\_mean),}
    \AttributeTok{mean\_mean =} \FunctionTok{mean}\NormalTok{(samples\_mean),}
    \AttributeTok{min\_error =} \FunctionTok{min}\NormalTok{(samples\_error),}
    \AttributeTok{max\_error =} \FunctionTok{max}\NormalTok{(samples\_error),}
    \AttributeTok{mean\_error =} \FunctionTok{mean}\NormalTok{(samples\_error)}
\NormalTok{  ) }\SpecialCharTok{\%\textgreater{}\%}
  \FunctionTok{select}\NormalTok{(}\SpecialCharTok{{-}}\NormalTok{samples, }\SpecialCharTok{{-}}\NormalTok{samples\_mean, }\SpecialCharTok{{-}}\NormalTok{samples\_error, }\SpecialCharTok{{-}}\NormalTok{leads)}

\NormalTok{sample\_mean\_means }\OtherTok{\textless{}{-}} \FunctionTok{mean}\NormalTok{(df\_leads\_daily\_error}\SpecialCharTok{$}\NormalTok{mean\_mean)}
\NormalTok{sample\_mean\_errors }\OtherTok{\textless{}{-}} \FunctionTok{mean}\NormalTok{(df\_leads\_daily\_error}\SpecialCharTok{$}\NormalTok{mean\_error)}

\NormalTok{colnames\_lookup }\OtherTok{\textless{}{-}} \FunctionTok{c}\NormalTok{(}
  \StringTok{"Média real"} \OtherTok{=} \StringTok{"lead\_mean"}\NormalTok{,}
  \StringTok{"Média das médias amostrais"} \OtherTok{=} \StringTok{"mean\_mean"}\NormalTok{,}
  \StringTok{"Erro médio"} \OtherTok{=} \StringTok{"mean\_error"}
\NormalTok{)}
\FunctionTok{sumtable}\NormalTok{(df\_leads\_daily\_error }\SpecialCharTok{\%\textgreater{}\%}
           \FunctionTok{select}\NormalTok{(lead\_mean, mean\_mean, mean\_error) }\SpecialCharTok{\%\textgreater{}\%}
           \FunctionTok{rename}\NormalTok{(}\FunctionTok{all\_of}\NormalTok{(colnames\_lookup)),}
         \AttributeTok{add.median =}\NormalTok{ T, }\AttributeTok{title =} \StringTok{"Erro da média de leads diários"}\NormalTok{)}
\end{Highlighting}
\end{Shaded}

\begin{table}

\caption{\label{tab:unnamed-chunk-13}Erro da média de leads diários}
\centering
\begin{tabular}[t]{lllllllll}
\toprule
Variable & N & Mean & Std. Dev. & Min & Pctl. 25 & Pctl. 50 & Pctl. 75 & Max\\
\midrule
Média real & 171 & 0.12 & 0.069 & 0 & 0.068 & 0.11 & 0.16 & 0.35\\
Média das médias amostrais & 171 & 0.12 & 0.083 & 0 & 0.062 & 0.1 & 0.16 & 0.51\\
Erro médio & 171 & 0.13 & 0.049 & 0 & 0.099 & 0.13 & 0.15 & 0.37\\
\bottomrule
\end{tabular}
\end{table}

Olhando a tabela acima, vemos que, com uma amostra de \texttt{4}
anúncios, a média das médias amostrais de leads diários é de
\texttt{0.12} leads, com um erro médio de \texttt{0.127} leads.

Ou seja, o valor é bem próximo da média real, no entanto, o erro médio é
bem maior \texttt{0.05}.

Mas pera aí\ldots{}

\includegraphics{./images/flork-cafe.jpg}

\begin{Shaded}
\begin{Highlighting}[]
\FunctionTok{grid.arrange}\NormalTok{(}
\NormalTok{  df\_leads\_daily\_error }\SpecialCharTok{\%\textgreater{}\%}
    \FunctionTok{ggplot}\NormalTok{(}\FunctionTok{aes}\NormalTok{(data\_criado\_em, mean\_error)) }\SpecialCharTok{+}
    \FunctionTok{geom\_line}\NormalTok{(}\AttributeTok{color =} \StringTok{"royalblue"}\NormalTok{, }\AttributeTok{linewidth =} \FloatTok{0.5}\NormalTok{) }\SpecialCharTok{+}
    \FunctionTok{geom\_hline}\NormalTok{(}\AttributeTok{yintercept =}\NormalTok{ max\_error, }\AttributeTok{color =} \StringTok{"red"}\NormalTok{, }\AttributeTok{linetype =} \StringTok{"dashed"}\NormalTok{, }\AttributeTok{alpha =} \FloatTok{0.5}\NormalTok{) }\SpecialCharTok{+}
    \FunctionTok{labs}\NormalTok{(}\AttributeTok{title =} \StringTok{"Erro absoluto da média de leads diários"}\NormalTok{,}
         \AttributeTok{x =} \StringTok{"Dia"}\NormalTok{,}
         \AttributeTok{y =} \StringTok{"Erro abs."}\NormalTok{) }\SpecialCharTok{+}
    \FunctionTok{apply.theme.ts}\NormalTok{(),}
\NormalTok{  df\_leads\_daily\_error }\SpecialCharTok{\%\textgreater{}\%}
    \FunctionTok{ggplot}\NormalTok{(}\FunctionTok{aes}\NormalTok{(mean\_error)) }\SpecialCharTok{+}
    \FunctionTok{coord\_cartesian}\NormalTok{(}\AttributeTok{xlim =} \FunctionTok{c}\NormalTok{(}\SpecialCharTok{{-}}\FloatTok{0.04}\NormalTok{, }\FloatTok{0.3}\NormalTok{)) }\SpecialCharTok{+}
    \FunctionTok{geom\_histogram}\NormalTok{(}\AttributeTok{bins =} \DecValTok{20}\NormalTok{, }\AttributeTok{color =} \FunctionTok{adjustcolor}\NormalTok{(}\StringTok{"royalblue"}\NormalTok{, }\AttributeTok{alpha.f =} \FloatTok{0.3}\NormalTok{), }\AttributeTok{fill =} \StringTok{"royalblue"}\NormalTok{, }\AttributeTok{alpha =} \FloatTok{0.5}\NormalTok{) }\SpecialCharTok{+}
    \FunctionTok{labs}\NormalTok{(}\AttributeTok{title =} \StringTok{""}\NormalTok{,}
         \AttributeTok{x =} \StringTok{""}\NormalTok{,}
         \AttributeTok{y =} \StringTok{""}\NormalTok{) }\SpecialCharTok{+}
\NormalTok{    theme.base }\SpecialCharTok{+}\NormalTok{ theme.no\_legend }\SpecialCharTok{+}\NormalTok{ theme.no\_axis }\SpecialCharTok{+}
    \FunctionTok{theme}\NormalTok{(}\AttributeTok{panel.grid.minor.y =} \FunctionTok{element\_blank}\NormalTok{()),}
\NormalTok{  df\_leads\_daily\_error }\SpecialCharTok{\%\textgreater{}\%}
    \FunctionTok{ggplot}\NormalTok{(}\FunctionTok{aes}\NormalTok{(mean\_error)) }\SpecialCharTok{+}
    \FunctionTok{coord\_cartesian}\NormalTok{(}\AttributeTok{xlim =} \FunctionTok{c}\NormalTok{(}\SpecialCharTok{{-}}\FloatTok{0.05}\NormalTok{, }\FloatTok{0.3}\NormalTok{)) }\SpecialCharTok{+}
    \FunctionTok{geom\_boxplot}\NormalTok{(}\AttributeTok{color =} \FunctionTok{adjustcolor}\NormalTok{(}\StringTok{"royalblue"}\NormalTok{, }\AttributeTok{alpha.f =} \FloatTok{0.8}\NormalTok{), }\AttributeTok{fill =} \StringTok{"royalblue"}\NormalTok{, }\AttributeTok{alpha =} \FloatTok{0.5}\NormalTok{) }\SpecialCharTok{+}
    \FunctionTok{labs}\NormalTok{(}\AttributeTok{title =} \StringTok{""}\NormalTok{,}
         \AttributeTok{x =} \StringTok{""}\NormalTok{,}
         \AttributeTok{y =} \StringTok{""}\NormalTok{) }\SpecialCharTok{+}
\NormalTok{    theme.base }\SpecialCharTok{+}\NormalTok{ theme.no\_legend }\SpecialCharTok{+}\NormalTok{ theme.no\_axis }\SpecialCharTok{+}
    \FunctionTok{theme}\NormalTok{(}\AttributeTok{axis.text.y =} \FunctionTok{element\_blank}\NormalTok{(),}
          \AttributeTok{axis.ticks.y =} \FunctionTok{element\_blank}\NormalTok{(),}
          \AttributeTok{panel.grid.major.y =} \FunctionTok{element\_blank}\NormalTok{(),}
          \AttributeTok{panel.grid.minor.y =} \FunctionTok{element\_blank}\NormalTok{()),}
  \AttributeTok{nrow =} \DecValTok{3}\NormalTok{,}
  \AttributeTok{heights =} \FunctionTok{c}\NormalTok{(}\DecValTok{3}\NormalTok{, }\DecValTok{2}\NormalTok{, }\FloatTok{1.5}\NormalTok{)}
\NormalTok{)}
\end{Highlighting}
\end{Shaded}

\begin{center} \ifthenelse{ \equal{width=\textwidth,height=\textheight,keepaspectratio=true}{} } { \includegraphics[width=2.5cm,height=2.5cm,keepaspectratio=true]{tamanho-amostra_files/figure-latex/unnamed-chunk-14-1} } { \includegraphics[width=\textwidth,height=\textheight,keepaspectratio=true]{tamanho-amostra_files/figure-latex/unnamed-chunk-14-1} }  \end{center}

Apesar de parecer correto quando analizamos o todo, o gráfico nos mostra
que estamos errando além do esperado a maior parte das vezes. Isso
indica que a amostra não é suficiente para garantir a precisão desejada.

\begin{Shaded}
\begin{Highlighting}[]
\NormalTok{df\_leads\_daily\_error }\SpecialCharTok{\%\textgreater{}\%}
  \FunctionTok{ggplot}\NormalTok{(}\FunctionTok{aes}\NormalTok{(data\_criado\_em)) }\SpecialCharTok{+}
  \FunctionTok{geom\_line}\NormalTok{(}\FunctionTok{aes}\NormalTok{(}\AttributeTok{y =}\NormalTok{ mean\_mean, }\AttributeTok{size =} \StringTok{"Estimado"}\NormalTok{), }\AttributeTok{color =} \StringTok{"blue"}\NormalTok{, }\AttributeTok{alpha =} \FloatTok{0.8}\NormalTok{, }\AttributeTok{linewidth =} \FloatTok{0.8}\NormalTok{) }\SpecialCharTok{+}
  \FunctionTok{geom\_line}\NormalTok{(}\FunctionTok{aes}\NormalTok{(}\AttributeTok{y =}\NormalTok{ lead\_mean, }\AttributeTok{size =} \StringTok{"Real"}\NormalTok{), }\AttributeTok{color =} \StringTok{"red"}\NormalTok{, }\AttributeTok{linetype =} \StringTok{"dashed"}\NormalTok{, }\AttributeTok{linewidth =} \FloatTok{0.8}\NormalTok{, }\AttributeTok{alpha =} \FloatTok{0.5}\NormalTok{) }\SpecialCharTok{+}
  \FunctionTok{labs}\NormalTok{(}\AttributeTok{title =} \StringTok{"Média de leads diários"}\NormalTok{,}
       \AttributeTok{x =} \StringTok{"Dia"}\NormalTok{,}
       \AttributeTok{y =} \StringTok{"Leads"}\NormalTok{,}
       \AttributeTok{size =} \StringTok{""}\NormalTok{) }\SpecialCharTok{+}
  \FunctionTok{apply.theme.ts.legend}\NormalTok{()}
\end{Highlighting}
\end{Shaded}

\begin{center} \ifthenelse{ \equal{width=\textwidth,height=\textheight,keepaspectratio=true}{} } { \includegraphics[width=2.5cm,height=2.5cm,keepaspectratio=true]{tamanho-amostra_files/figure-latex/unnamed-chunk-15-1} } { \includegraphics[width=\textwidth,height=\textheight,keepaspectratio=true]{tamanho-amostra_files/figure-latex/unnamed-chunk-15-1} }  \end{center}

Mas então, o que pode ter dado errado?

\subsubsection{Causas do erro}\label{causas-do-erro}

\begin{itemize}
\tightlist
\item
  \textbf{Variabilidade dos dados}: para calcular o tamanho da amostra
  utilizamos a média de todas as médias diárias, desta forma, perdemos a
  informação de que há dias com alta variabilidade.
\item
  \textbf{Independência}: outro fator que influencia nesse erro é a
  falta de independência entre os dados. Isto é, os dados utilizados
  formam uma série temporal, o que faz com que qualquer estatística
  utilizada seja dependente do tempo. Isso torna incorreta a técnica de
  estimação do tamanho da amostra, pois tem como premissa a
  independência dos dados.
\end{itemize}

\subsection{Sugestão de tamanho da
amostra}\label{sugestuxe3o-de-tamanho-da-amostra}

Para trabalhos futuros sugerimos que \textbf{não} utilize as técnicas de
estimação de tamanho de amostra apresentadas neste trabalho. Em vez
disso, sugerimos que sejam utilizadas técnicas de amostragem próprias
para séries temporais.

\includegraphics{./images/flork-harmonia.png}

\end{document}
