% Options for packages loaded elsewhere
\PassOptionsToPackage{unicode}{hyperref}
\PassOptionsToPackage{hyphens}{url}
%
\documentclass[
]{article}
\usepackage{amsmath,amssymb}
\usepackage{iftex}
\ifPDFTeX
  \usepackage[T1]{fontenc}
  \usepackage[utf8]{inputenc}
  \usepackage{textcomp} % provide euro and other symbols
\else % if luatex or xetex
  \usepackage{unicode-math} % this also loads fontspec
  \defaultfontfeatures{Scale=MatchLowercase}
  \defaultfontfeatures[\rmfamily]{Ligatures=TeX,Scale=1}
\fi
\usepackage{lmodern}
\ifPDFTeX\else
  % xetex/luatex font selection
\fi
% Use upquote if available, for straight quotes in verbatim environments
\IfFileExists{upquote.sty}{\usepackage{upquote}}{}
\IfFileExists{microtype.sty}{% use microtype if available
  \usepackage[]{microtype}
  \UseMicrotypeSet[protrusion]{basicmath} % disable protrusion for tt fonts
}{}
\makeatletter
\@ifundefined{KOMAClassName}{% if non-KOMA class
  \IfFileExists{parskip.sty}{%
    \usepackage{parskip}
  }{% else
    \setlength{\parindent}{0pt}
    \setlength{\parskip}{6pt plus 2pt minus 1pt}}
}{% if KOMA class
  \KOMAoptions{parskip=half}}
\makeatother
\usepackage{xcolor}
\usepackage[margin=1in]{geometry}
\usepackage{color}
\usepackage{fancyvrb}
\newcommand{\VerbBar}{|}
\newcommand{\VERB}{\Verb[commandchars=\\\{\}]}
\DefineVerbatimEnvironment{Highlighting}{Verbatim}{commandchars=\\\{\}}
% Add ',fontsize=\small' for more characters per line
\usepackage{framed}
\definecolor{shadecolor}{RGB}{248,248,248}
\newenvironment{Shaded}{\begin{snugshade}}{\end{snugshade}}
\newcommand{\AlertTok}[1]{\textcolor[rgb]{0.94,0.16,0.16}{#1}}
\newcommand{\AnnotationTok}[1]{\textcolor[rgb]{0.56,0.35,0.01}{\textbf{\textit{#1}}}}
\newcommand{\AttributeTok}[1]{\textcolor[rgb]{0.13,0.29,0.53}{#1}}
\newcommand{\BaseNTok}[1]{\textcolor[rgb]{0.00,0.00,0.81}{#1}}
\newcommand{\BuiltInTok}[1]{#1}
\newcommand{\CharTok}[1]{\textcolor[rgb]{0.31,0.60,0.02}{#1}}
\newcommand{\CommentTok}[1]{\textcolor[rgb]{0.56,0.35,0.01}{\textit{#1}}}
\newcommand{\CommentVarTok}[1]{\textcolor[rgb]{0.56,0.35,0.01}{\textbf{\textit{#1}}}}
\newcommand{\ConstantTok}[1]{\textcolor[rgb]{0.56,0.35,0.01}{#1}}
\newcommand{\ControlFlowTok}[1]{\textcolor[rgb]{0.13,0.29,0.53}{\textbf{#1}}}
\newcommand{\DataTypeTok}[1]{\textcolor[rgb]{0.13,0.29,0.53}{#1}}
\newcommand{\DecValTok}[1]{\textcolor[rgb]{0.00,0.00,0.81}{#1}}
\newcommand{\DocumentationTok}[1]{\textcolor[rgb]{0.56,0.35,0.01}{\textbf{\textit{#1}}}}
\newcommand{\ErrorTok}[1]{\textcolor[rgb]{0.64,0.00,0.00}{\textbf{#1}}}
\newcommand{\ExtensionTok}[1]{#1}
\newcommand{\FloatTok}[1]{\textcolor[rgb]{0.00,0.00,0.81}{#1}}
\newcommand{\FunctionTok}[1]{\textcolor[rgb]{0.13,0.29,0.53}{\textbf{#1}}}
\newcommand{\ImportTok}[1]{#1}
\newcommand{\InformationTok}[1]{\textcolor[rgb]{0.56,0.35,0.01}{\textbf{\textit{#1}}}}
\newcommand{\KeywordTok}[1]{\textcolor[rgb]{0.13,0.29,0.53}{\textbf{#1}}}
\newcommand{\NormalTok}[1]{#1}
\newcommand{\OperatorTok}[1]{\textcolor[rgb]{0.81,0.36,0.00}{\textbf{#1}}}
\newcommand{\OtherTok}[1]{\textcolor[rgb]{0.56,0.35,0.01}{#1}}
\newcommand{\PreprocessorTok}[1]{\textcolor[rgb]{0.56,0.35,0.01}{\textit{#1}}}
\newcommand{\RegionMarkerTok}[1]{#1}
\newcommand{\SpecialCharTok}[1]{\textcolor[rgb]{0.81,0.36,0.00}{\textbf{#1}}}
\newcommand{\SpecialStringTok}[1]{\textcolor[rgb]{0.31,0.60,0.02}{#1}}
\newcommand{\StringTok}[1]{\textcolor[rgb]{0.31,0.60,0.02}{#1}}
\newcommand{\VariableTok}[1]{\textcolor[rgb]{0.00,0.00,0.00}{#1}}
\newcommand{\VerbatimStringTok}[1]{\textcolor[rgb]{0.31,0.60,0.02}{#1}}
\newcommand{\WarningTok}[1]{\textcolor[rgb]{0.56,0.35,0.01}{\textbf{\textit{#1}}}}
\usepackage{longtable,booktabs,array}
\usepackage{calc} % for calculating minipage widths
% Correct order of tables after \paragraph or \subparagraph
\usepackage{etoolbox}
\makeatletter
\patchcmd\longtable{\par}{\if@noskipsec\mbox{}\fi\par}{}{}
\makeatother
% Allow footnotes in longtable head/foot
\IfFileExists{footnotehyper.sty}{\usepackage{footnotehyper}}{\usepackage{footnote}}
\makesavenoteenv{longtable}
\usepackage{graphicx}
\makeatletter
\def\maxwidth{\ifdim\Gin@nat@width>\linewidth\linewidth\else\Gin@nat@width\fi}
\def\maxheight{\ifdim\Gin@nat@height>\textheight\textheight\else\Gin@nat@height\fi}
\makeatother
% Scale images if necessary, so that they will not overflow the page
% margins by default, and it is still possible to overwrite the defaults
% using explicit options in \includegraphics[width, height, ...]{}
\setkeys{Gin}{width=\maxwidth,height=\maxheight,keepaspectratio}
% Set default figure placement to htbp
\makeatletter
\def\fps@figure{htbp}
\makeatother
\setlength{\emergencystretch}{3em} % prevent overfull lines
\providecommand{\tightlist}{%
  \setlength{\itemsep}{0pt}\setlength{\parskip}{0pt}}
\setcounter{secnumdepth}{-\maxdimen} % remove section numbering
\usepackage{ifthen}
\let\oldincludegraphics\includegraphics
\renewcommand{\includegraphics}[2][]{ \ifthenelse{ \equal{#1}{} } { \oldincludegraphics[width=2.5cm,height=2.5cm,keepaspectratio=true]{#2} } { \oldincludegraphics[#1]{#2} } }
\ifLuaTeX
  \usepackage{selnolig}  % disable illegal ligatures
\fi
\usepackage{bookmark}
\IfFileExists{xurl.sty}{\usepackage{xurl}}{} % add URL line breaks if available
\urlstyle{same}
\hypersetup{
  pdftitle={Tamanho mínimo de amostra},
  pdfauthor={Arquimedes Macedo. Tiago Rodrigues},
  hidelinks,
  pdfcreator={LaTeX via pandoc}}

\title{Tamanho mínimo de amostra}
\author{Arquimedes Macedo. Tiago Rodrigues}
\date{}

\begin{document}
\maketitle

{
\setcounter{tocdepth}{2}
\tableofcontents
}
\begin{Shaded}
\begin{Highlighting}[]
\FunctionTok{library}\NormalTok{(dplyr)}
\FunctionTok{library}\NormalTok{(tidyr)}
\FunctionTok{library}\NormalTok{(readxl)}
\FunctionTok{library}\NormalTok{(knitr)}
\FunctionTok{library}\NormalTok{(ggplot2)}
\FunctionTok{library}\NormalTok{(ggridges)}
\FunctionTok{library}\NormalTok{(reshape2)}
\FunctionTok{library}\NormalTok{(gridExtra)}
\FunctionTok{library}\NormalTok{(vtable)}
\FunctionTok{library}\NormalTok{(purrr)}

\CommentTok{\# Centering figures chunk output}
\NormalTok{knitr}\SpecialCharTok{::}\NormalTok{opts\_chunk}\SpecialCharTok{$}\FunctionTok{set}\NormalTok{(}\AttributeTok{out.height =} \StringTok{"}\SpecialCharTok{\textbackslash{}\textbackslash{}}\StringTok{textheight"}\NormalTok{,  }\AttributeTok{out.width =} \StringTok{"}\SpecialCharTok{\textbackslash{}\textbackslash{}}\StringTok{textwidth"}\NormalTok{,}
                      \AttributeTok{out.extra =} \StringTok{"keepaspectratio=true"}\NormalTok{, }\AttributeTok{fig.align =} \StringTok{"center"}\NormalTok{)}
\end{Highlighting}
\end{Shaded}

\begin{Shaded}
\begin{Highlighting}[]
\NormalTok{theme.base }\OtherTok{\textless{}{-}} \FunctionTok{theme\_minimal}\NormalTok{(}\AttributeTok{base\_size =} \DecValTok{11}\NormalTok{) }\SpecialCharTok{+}
  \FunctionTok{theme}\NormalTok{(}
    \AttributeTok{axis.text =} \FunctionTok{element\_text}\NormalTok{(}\AttributeTok{size =} \DecValTok{8}\NormalTok{),}
    \AttributeTok{plot.title =} \FunctionTok{element\_text}\NormalTok{(}\AttributeTok{hjust =} \FloatTok{0.5}\NormalTok{, }\AttributeTok{size =} \DecValTok{12}\NormalTok{),}
    \AttributeTok{axis.title =} \FunctionTok{element\_text}\NormalTok{(}\AttributeTok{size =} \DecValTok{10}\NormalTok{),}
    \AttributeTok{panel.grid.major =} \FunctionTok{element\_line}\NormalTok{(}\AttributeTok{colour =} \StringTok{"grey90"}\NormalTok{, }\AttributeTok{linewidth =} \FloatTok{0.5}\NormalTok{),}
    \AttributeTok{panel.grid.minor =} \FunctionTok{element\_line}\NormalTok{(}\AttributeTok{colour =} \FunctionTok{adjustcolor}\NormalTok{(}\StringTok{"grey90"}\NormalTok{, }\AttributeTok{alpha.f =} \FloatTok{0.5}\NormalTok{), }\AttributeTok{linewidth =} \FloatTok{0.25}\NormalTok{),}
    \AttributeTok{panel.border =} \FunctionTok{element\_blank}\NormalTok{(),}
    \AttributeTok{panel.background =} \FunctionTok{element\_blank}\NormalTok{(),}
    \AttributeTok{plot.background =} \FunctionTok{element\_blank}\NormalTok{(),}
    \AttributeTok{axis.line.x =} \FunctionTok{element\_line}\NormalTok{(}\AttributeTok{colour =} \StringTok{"grey"}\NormalTok{),}
    \AttributeTok{axis.line.y =} \FunctionTok{element\_line}\NormalTok{(}\AttributeTok{colour =} \StringTok{"grey"}\NormalTok{),}
\NormalTok{  )}

\NormalTok{theme.no\_legend }\OtherTok{\textless{}{-}} \FunctionTok{theme}\NormalTok{(}\AttributeTok{legend.position =} \StringTok{"none"}\NormalTok{)}

\NormalTok{theme.no\_grid }\OtherTok{\textless{}{-}}  \FunctionTok{theme}\NormalTok{(}
  \AttributeTok{panel.grid.major =} \FunctionTok{element\_blank}\NormalTok{(),}
  \AttributeTok{panel.grid.minor =} \FunctionTok{element\_blank}\NormalTok{()}
\NormalTok{)}

\NormalTok{theme.no\_axis }\OtherTok{\textless{}{-}} \FunctionTok{theme}\NormalTok{(}
  \AttributeTok{axis.line.x =} \FunctionTok{element\_blank}\NormalTok{(),}
  \AttributeTok{axis.line.y =} \FunctionTok{element\_blank}\NormalTok{()}
\NormalTok{)}

\NormalTok{apply.theme.ts }\OtherTok{\textless{}{-}} \ControlFlowTok{function}\NormalTok{() \{}
  \FunctionTok{list}\NormalTok{(}
    \FunctionTok{scale\_x\_date}\NormalTok{(}\AttributeTok{date\_labels =} \StringTok{"\%b \%d"}\NormalTok{, }\AttributeTok{date\_breaks =} \StringTok{"1 week"}\NormalTok{),}
\NormalTok{    theme.base }\SpecialCharTok{+}\NormalTok{ theme.no\_legend }\SpecialCharTok{+}
      \FunctionTok{theme}\NormalTok{(}
        \AttributeTok{axis.text.x =} \FunctionTok{element\_text}\NormalTok{(}\AttributeTok{angle =} \DecValTok{45}\NormalTok{, }\AttributeTok{hjust =} \DecValTok{1}\NormalTok{),}
        \AttributeTok{panel.grid.major.x =} \FunctionTok{element\_blank}\NormalTok{(),}
        \AttributeTok{panel.grid.minor.x =} \FunctionTok{element\_blank}\NormalTok{()}
\NormalTok{      )}
\NormalTok{  )}
\NormalTok{\}}
\end{Highlighting}
\end{Shaded}

\subsection{Objetivo}\label{objetivo}

Estimar a quantidade média de leads diários, com 80\% de confiança, por
anunciante, de anúncios de vendas de imóveis na cidade de Florianópolis
(SC).

Com um erro máximo de 0.05, usando Amostragem Aleatório Simples sem
Reposição (AASs).

\emph{Lead}: é um contato de um cliente em potencial que demonstrou
interesse em um produto ou serviço.

IC de 80\% foi escolhido devido à falta de informações (descrita logo
mais), e também por ser este o valor máximo recomendado pela ABNT para
avaliações de imóveis (NBR 14653).

\subsection{Metodologia}\label{metodologia}

Almeja-se, a partir de uma lista de anúncios, realizar uma busca diária
de leads, usando uma amostra dos anúncios, e, a partir destes dados,
estimar a quantidade média de leads.

No entanto, entende-se que há limitações nas informações disponíveis,
como:

\begin{itemize}
\tightlist
\item
  O número de leads por anúncio.
\item
  Tempo total que o anúncio ficou ativo.
\item
  A sazonalidade do mercado (oferta e demanda).
\item
  A eficácia do anúncio (qualidade do anúncio, preço, localização, etc).
\item
  A qualidade dos leads (interesse real ou apenas curiosidade).
\item
  A distribuição subjacente dos leads ao longo do tempo.
\end{itemize}

Desta forma como um estudo piloto, foram obtidos leads diários, entre
Janeiro e Julho de 2024, de anúncios de um anunciante na cidade alvo.

\subsubsection{Análise exploratória}\label{anuxe1lise-exploratuxf3ria}

O banco de dados é composto por 3 colunas:

\begin{itemize}
\tightlist
\item
  \texttt{id\_registro}: identificador do lead.
\item
  \texttt{data\_criado\_em}: dia que o lead foi gerado.
\item
  \texttt{id\_anuncio}: identificador do anúncio.
\end{itemize}

\paragraph{Amostra dos dados}\label{amostra-dos-dados}

\begin{Shaded}
\begin{Highlighting}[]
\NormalTok{df\_leads }\OtherTok{\textless{}{-}} \FunctionTok{read\_excel}\NormalTok{(}\StringTok{"dataset/leads.xlsx"}\NormalTok{, }\AttributeTok{col\_types =} \FunctionTok{c}\NormalTok{(}\StringTok{"numeric"}\NormalTok{, }\StringTok{"date"}\NormalTok{, }\StringTok{"text"}\NormalTok{))}
\NormalTok{df\_leads}\SpecialCharTok{$}\NormalTok{data\_criado\_em }\OtherTok{\textless{}{-}} \FunctionTok{as.Date}\NormalTok{(df\_leads}\SpecialCharTok{$}\NormalTok{data\_criado\_em)}
\FunctionTok{kable}\NormalTok{(}\FunctionTok{head}\NormalTok{(df\_leads))}
\end{Highlighting}
\end{Shaded}

\begin{longtable}[]{@{}rll@{}}
\toprule\noalign{}
id\_registro & data\_criado\_em & id\_anuncio \\
\midrule\noalign{}
\endhead
\bottomrule\noalign{}
\endlastfoot
1 & 2024-01-18 & LRB3GK \\
2 & 2024-01-19 & 4I931S \\
3 & 2024-01-19 & 4WUWGH \\
4 & 2024-01-19 & XNI94R \\
5 & 2024-01-20 & HRDJQG \\
6 & 2024-01-20 & CH8NIW \\
\end{longtable}

\paragraph{Leads diários}\label{leads-diuxe1rios}

\begin{Shaded}
\begin{Highlighting}[]
\NormalTok{df\_leads }\SpecialCharTok{\%\textgreater{}\%}
  \FunctionTok{group\_by}\NormalTok{(data\_criado\_em) }\SpecialCharTok{\%\textgreater{}\%}
  \FunctionTok{summarise}\NormalTok{(}\AttributeTok{leads =} \FunctionTok{n}\NormalTok{()) }\SpecialCharTok{\%\textgreater{}\%}
  \FunctionTok{ggplot}\NormalTok{(}\FunctionTok{aes}\NormalTok{(data\_criado\_em, leads)) }\SpecialCharTok{+}
  \FunctionTok{geom\_line}\NormalTok{(}\AttributeTok{color =} \StringTok{"royalblue"}\NormalTok{, }\AttributeTok{linewidth =} \FloatTok{0.5}\NormalTok{) }\SpecialCharTok{+}
  \FunctionTok{labs}\NormalTok{(}\AttributeTok{title =} \StringTok{"Leads diários"}\NormalTok{,}
       \AttributeTok{x =} \StringTok{"Dia"}\NormalTok{,}
       \AttributeTok{y =} \StringTok{"Leads"}\NormalTok{) }\SpecialCharTok{+}
  \FunctionTok{apply.theme.ts}\NormalTok{()}
\end{Highlighting}
\end{Shaded}

\begin{center} \ifthenelse{ \equal{width=\textwidth,height=\textheight,keepaspectratio=true}{} } { \includegraphics[width=2.5cm,height=2.5cm,keepaspectratio=true]{tamanho-amostra_files/figure-latex/unnamed-chunk-4-1} } { \includegraphics[width=\textwidth,height=\textheight,keepaspectratio=true]{tamanho-amostra_files/figure-latex/unnamed-chunk-4-1} }  \end{center}

Notam-se picos em intervalos semi-regulares, o que pode indicar
sazonalidade ou eventos específicos. Além disso, em Julho, houve uma
alta variabilidade nos leads diários.

\paragraph{Leads por anúncio}\label{leads-por-anuxfancio}

Vamos analisar a média diária de leads por anúncio.

Para isso, dividimos o número total de leads pelo número de anúncios
únicos para cada dia.

\begin{Shaded}
\begin{Highlighting}[]
\NormalTok{df\_leads\_incorrect\_mean }\OtherTok{\textless{}{-}}\NormalTok{ df\_leads }\SpecialCharTok{\%\textgreater{}\%}
  \FunctionTok{group\_by}\NormalTok{(data\_criado\_em) }\SpecialCharTok{\%\textgreater{}\%}
  \FunctionTok{summarise}\NormalTok{(}\AttributeTok{mean =} \FunctionTok{n}\NormalTok{()}\SpecialCharTok{/}\FunctionTok{length}\NormalTok{(}\FunctionTok{unique}\NormalTok{(id\_anuncio)))}

\FunctionTok{grid.arrange}\NormalTok{(}
\NormalTok{  df\_leads\_incorrect\_mean }\SpecialCharTok{\%\textgreater{}\%}
    \FunctionTok{ggplot}\NormalTok{(}\FunctionTok{aes}\NormalTok{(data\_criado\_em, mean)) }\SpecialCharTok{+}
    \FunctionTok{geom\_line}\NormalTok{(}\AttributeTok{color =} \StringTok{"royalblue"}\NormalTok{, }\AttributeTok{linewidth =} \FloatTok{0.5}\NormalTok{) }\SpecialCharTok{+}
    \FunctionTok{labs}\NormalTok{(}\AttributeTok{title =} \StringTok{"Média de leads por dia"}\NormalTok{,}
         \AttributeTok{x =} \StringTok{"Dia"}\NormalTok{,}
         \AttributeTok{y =} \StringTok{"Média de leads"}\NormalTok{) }\SpecialCharTok{+}
    \FunctionTok{apply.theme.ts}\NormalTok{(),}
\NormalTok{  df\_leads\_incorrect\_mean }\SpecialCharTok{\%\textgreater{}\%}
    \FunctionTok{ggplot}\NormalTok{(}\FunctionTok{aes}\NormalTok{(mean)) }\SpecialCharTok{+}
    \FunctionTok{geom\_histogram}\NormalTok{(}\AttributeTok{bins =} \DecValTok{30}\NormalTok{, }\AttributeTok{color =} \StringTok{"royalblue"}\NormalTok{, }\AttributeTok{fill =} \StringTok{"royalblue"}\NormalTok{, }\AttributeTok{alpha =} \FloatTok{0.5}\NormalTok{) }\SpecialCharTok{+}
    \FunctionTok{labs}\NormalTok{(}\AttributeTok{title =} \StringTok{""}\NormalTok{,}
         \AttributeTok{x =} \StringTok{"Média de leads"}\NormalTok{,}
         \AttributeTok{y =} \StringTok{""}\NormalTok{) }\SpecialCharTok{+}
\NormalTok{    theme.base }\SpecialCharTok{+}\NormalTok{ theme.no\_legend,}
  \AttributeTok{nrow =} \DecValTok{2}
\NormalTok{)}
\end{Highlighting}
\end{Shaded}

\begin{center} \ifthenelse{ \equal{width=\textwidth,height=\textheight,keepaspectratio=true}{} } { \includegraphics[width=2.5cm,height=2.5cm,keepaspectratio=true]{tamanho-amostra_files/figure-latex/unnamed-chunk-5-1} } { \includegraphics[width=\textwidth,height=\textheight,keepaspectratio=true]{tamanho-amostra_files/figure-latex/unnamed-chunk-5-1} }  \end{center}

Será que é só isso mesmo?

\includegraphics{./images/flork-pensando.jpg}

Claro que não! A média diária de leads por anúncio é uma estimativa
incorreta, pois não considera a quantidade de anúncios ativos em cada
dia, e acaba gerando um viés.

\subsubsection{Estimativa da média}\label{estimativa-da-muxe9dia}

Para corrigir o problema anterior, vamos completar os dados com zeros
para os dias sem leads.

Isto é, vamos pegar o primeiro e o último dia que o anúncio teve leads,
e criar novos registros entre estas datas, para dias sem lead.

\begin{Shaded}
\begin{Highlighting}[]
\NormalTok{df\_leads\_complete }\OtherTok{\textless{}{-}}\NormalTok{ df\_leads }\SpecialCharTok{\%\textgreater{}\%}
  \FunctionTok{group\_by}\NormalTok{(id\_anuncio, data\_criado\_em) }\SpecialCharTok{\%\textgreater{}\%}
  \FunctionTok{summarise}\NormalTok{(}\AttributeTok{leads =} \FunctionTok{n}\NormalTok{(), }\AttributeTok{.groups =} \StringTok{\textquotesingle{}drop\textquotesingle{}}\NormalTok{) }\SpecialCharTok{\%\textgreater{}\%}
  \FunctionTok{group\_by}\NormalTok{(id\_anuncio) }\SpecialCharTok{\%\textgreater{}\%}
  \CommentTok{\# Creates a list of dataframes by id}
\NormalTok{  tidyr}\SpecialCharTok{::}\FunctionTok{nest}\NormalTok{() }\SpecialCharTok{\%\textgreater{}\%}
  \FunctionTok{mutate}\NormalTok{(}
    \CommentTok{\# Creates a sequence of dates by id}
    \AttributeTok{date\_seq =} \FunctionTok{map}\NormalTok{(data, }\SpecialCharTok{\textasciitilde{}}\FunctionTok{seq}\NormalTok{(}\FunctionTok{min}\NormalTok{(.}\SpecialCharTok{$}\NormalTok{data\_criado\_em), }\FunctionTok{max}\NormalTok{(.}\SpecialCharTok{$}\NormalTok{data\_criado\_em), }\AttributeTok{by =} \StringTok{"day"}\NormalTok{)),}
    \CommentTok{\# Completes the missing dates}
    \AttributeTok{data =} \FunctionTok{map2}\NormalTok{(}
\NormalTok{      data, date\_seq,}
\NormalTok{      \textbackslash{}(data\_, seq\_) \{}
\NormalTok{        data\_ }\SpecialCharTok{\%\textgreater{}\%}
          \FunctionTok{complete}\NormalTok{(}\AttributeTok{data\_criado\_em =}\NormalTok{ seq\_, }\AttributeTok{fill =} \FunctionTok{list}\NormalTok{(}\AttributeTok{leads =} \DecValTok{0}\NormalTok{))}
\NormalTok{      \}}
\NormalTok{    )}
\NormalTok{  ) }\SpecialCharTok{\%\textgreater{}\%}
  \CommentTok{\# Removes the auxiliary column}
  \FunctionTok{select}\NormalTok{(}\SpecialCharTok{{-}}\NormalTok{date\_seq) }\SpecialCharTok{\%\textgreater{}\%}
  \CommentTok{\# Unnests the data}
  \FunctionTok{unnest}\NormalTok{(data)}

\FunctionTok{kable}\NormalTok{(}\FunctionTok{head}\NormalTok{(df\_leads\_complete))}
\end{Highlighting}
\end{Shaded}

\begin{longtable}[]{@{}llr@{}}
\toprule\noalign{}
id\_anuncio & data\_criado\_em & leads \\
\midrule\noalign{}
\endhead
\bottomrule\noalign{}
\endlastfoot
00OPP2 & 2024-03-10 & 1 \\
00SLR7 & 2024-06-29 & 1 \\
00TPRF & 2024-06-23 & 1 \\
02NTL4 & 2024-04-20 & 1 \\
02NTL4 & 2024-04-21 & 0 \\
02NTL4 & 2024-04-22 & 0 \\
\end{longtable}

A partir desta correção, temos as seguintes médias diárias.

\begin{Shaded}
\begin{Highlighting}[]
\NormalTok{df\_leads\_daily }\OtherTok{\textless{}{-}}\NormalTok{ df\_leads\_complete }\SpecialCharTok{\%\textgreater{}\%}
  \FunctionTok{group\_by}\NormalTok{(data\_criado\_em) }\SpecialCharTok{\%\textgreater{}\%}
  \FunctionTok{summarise}\NormalTok{(}\AttributeTok{mean =} \FunctionTok{mean}\NormalTok{(leads),}
            \AttributeTok{total\_leads =} \FunctionTok{sum}\NormalTok{(leads),}
            \AttributeTok{active\_listings =} \FunctionTok{n\_distinct}\NormalTok{(id\_anuncio))}

\FunctionTok{grid.arrange}\NormalTok{(}
\NormalTok{  df\_leads\_daily }\SpecialCharTok{\%\textgreater{}\%}
    \FunctionTok{ggplot}\NormalTok{(}\FunctionTok{aes}\NormalTok{(data\_criado\_em, mean)) }\SpecialCharTok{+}
    \FunctionTok{geom\_line}\NormalTok{(}\AttributeTok{color =} \StringTok{"royalblue"}\NormalTok{, }\AttributeTok{linewidth =} \FloatTok{0.5}\NormalTok{) }\SpecialCharTok{+}
    \FunctionTok{labs}\NormalTok{(}\AttributeTok{title =} \StringTok{"Média de leads por dia"}\NormalTok{,}
         \AttributeTok{x =} \StringTok{"Dia"}\NormalTok{,}
         \AttributeTok{y =} \StringTok{"Média de leads"}\NormalTok{) }\SpecialCharTok{+}
    \FunctionTok{apply.theme.ts}\NormalTok{(),}
\NormalTok{  df\_leads\_daily }\SpecialCharTok{\%\textgreater{}\%}
    \FunctionTok{ggplot}\NormalTok{(}\FunctionTok{aes}\NormalTok{(mean)) }\SpecialCharTok{+}
    \FunctionTok{geom\_histogram}\NormalTok{(}\AttributeTok{bins =} \DecValTok{20}\NormalTok{, }\AttributeTok{color =} \StringTok{"royalblue"}\NormalTok{, }\AttributeTok{fill =} \StringTok{"royalblue"}\NormalTok{, }\AttributeTok{alpha =} \FloatTok{0.5}\NormalTok{) }\SpecialCharTok{+}
    \FunctionTok{labs}\NormalTok{(}\AttributeTok{title =} \StringTok{""}\NormalTok{,}
         \AttributeTok{x =} \StringTok{"Média de leads"}\NormalTok{,}
         \AttributeTok{y =} \StringTok{""}\NormalTok{) }\SpecialCharTok{+}
\NormalTok{    theme.base }\SpecialCharTok{+}\NormalTok{ theme.no\_legend,}
  \AttributeTok{nrow =} \DecValTok{2}
\NormalTok{)}
\end{Highlighting}
\end{Shaded}

\begin{center} \ifthenelse{ \equal{width=\textwidth,height=\textheight,keepaspectratio=true}{} } { \includegraphics[width=2.5cm,height=2.5cm,keepaspectratio=true]{tamanho-amostra_files/figure-latex/unnamed-chunk-7-1} } { \includegraphics[width=\textwidth,height=\textheight,keepaspectratio=true]{tamanho-amostra_files/figure-latex/unnamed-chunk-7-1} }  \end{center}

Mas não está totalmente correto\ldots{}

\includegraphics{./images/flork-exercer-a-calma.jpg}

Lembrando que esta é uma aproximação e não corresponde totalmente ao que
de fato aconteceu, para computar a verdadeira média, precisariamos da
listagem de todos os anúncios ativos no dia.

Nota-se, também, que existem pontos extremos no início e no fim da
série, isso pode ser explicado por anúncios que estavam ativos antes do
início do período analisado ou que apareceram um pouco antes do fim.

Portanto vamos analisar apenas entre 01/02/2024 e 20/07/2024.

\begin{Shaded}
\begin{Highlighting}[]
\NormalTok{df\_leads\_complete\_filtered }\OtherTok{\textless{}{-}}\NormalTok{ df\_leads\_complete }\SpecialCharTok{\%\textgreater{}\%}
  \FunctionTok{filter}\NormalTok{(}\FunctionTok{between}\NormalTok{(data\_criado\_em, }\FunctionTok{as.Date}\NormalTok{(}\StringTok{"2024{-}02{-}01"}\NormalTok{), }\FunctionTok{as.Date}\NormalTok{(}\StringTok{"2024{-}07{-}20"}\NormalTok{)))}

\NormalTok{df\_leads\_daily\_filtered }\OtherTok{\textless{}{-}}\NormalTok{ df\_leads\_complete\_filtered }\SpecialCharTok{\%\textgreater{}\%}
  \FunctionTok{group\_by}\NormalTok{(data\_criado\_em) }\SpecialCharTok{\%\textgreater{}\%}
  \FunctionTok{summarise}\NormalTok{(}\AttributeTok{mean =} \FunctionTok{mean}\NormalTok{(leads),}
            \AttributeTok{total\_leads =} \FunctionTok{sum}\NormalTok{(leads),}
            \AttributeTok{active\_listings =} \FunctionTok{n\_distinct}\NormalTok{(id\_anuncio))}
\end{Highlighting}
\end{Shaded}

\begin{Shaded}
\begin{Highlighting}[]
\FunctionTok{sumtable}\NormalTok{(df\_leads\_complete\_filtered, }\AttributeTok{add.median =}\NormalTok{ T, }\AttributeTok{title =} \StringTok{"Registros corrigidos"}\NormalTok{)}
\end{Highlighting}
\end{Shaded}

\begin{table}

\caption{\label{tab:unnamed-chunk-9}Registros corrigidos}
\centering
\begin{tabular}[t]{lllllllll}
\toprule
Variable & N & Mean & Std. Dev. & Min & Pctl. 25 & Pctl. 50 & Pctl. 75 & Max\\
\midrule
leads & 11253 & 0.12 & 0.34 & 0 & 0 & 0 & 0 & 4\\
\bottomrule
\end{tabular}
\end{table}

\begin{Shaded}
\begin{Highlighting}[]
\FunctionTok{grid.arrange}\NormalTok{(}
\NormalTok{  df\_leads\_daily\_filtered }\SpecialCharTok{\%\textgreater{}\%}
    \FunctionTok{ggplot}\NormalTok{(}\FunctionTok{aes}\NormalTok{(data\_criado\_em, mean)) }\SpecialCharTok{+}
    \FunctionTok{geom\_line}\NormalTok{(}\AttributeTok{color =} \StringTok{"royalblue"}\NormalTok{, }\AttributeTok{linewidth =} \FloatTok{0.5}\NormalTok{) }\SpecialCharTok{+}
    \FunctionTok{labs}\NormalTok{(}\AttributeTok{title =} \StringTok{"Média de leads por dia"}\NormalTok{,}
         \AttributeTok{x =} \StringTok{"Dia"}\NormalTok{,}
         \AttributeTok{y =} \StringTok{"Média de leads"}\NormalTok{) }\SpecialCharTok{+}
    \FunctionTok{apply.theme.ts}\NormalTok{(),}
\NormalTok{  df\_leads\_daily\_filtered }\SpecialCharTok{\%\textgreater{}\%}
    \FunctionTok{ggplot}\NormalTok{(}\FunctionTok{aes}\NormalTok{(mean)) }\SpecialCharTok{+}
    \FunctionTok{coord\_cartesian}\NormalTok{(}\AttributeTok{xlim =} \FunctionTok{c}\NormalTok{(}\SpecialCharTok{{-}}\FloatTok{0.01}\NormalTok{, }\FloatTok{0.4}\NormalTok{)) }\SpecialCharTok{+}
    \FunctionTok{geom\_histogram}\NormalTok{(}\AttributeTok{bins =} \DecValTok{20}\NormalTok{, }\AttributeTok{color =} \FunctionTok{adjustcolor}\NormalTok{(}\StringTok{"royalblue"}\NormalTok{, }\AttributeTok{alpha.f =} \FloatTok{0.3}\NormalTok{), }\AttributeTok{fill =} \StringTok{"royalblue"}\NormalTok{, }\AttributeTok{alpha =} \FloatTok{0.5}\NormalTok{) }\SpecialCharTok{+}
    \FunctionTok{labs}\NormalTok{(}\AttributeTok{title =} \StringTok{""}\NormalTok{,}
         \AttributeTok{x =} \StringTok{""}\NormalTok{,}
         \AttributeTok{y =} \StringTok{""}\NormalTok{) }\SpecialCharTok{+}
\NormalTok{    theme.base }\SpecialCharTok{+}\NormalTok{ theme.no\_legend }\SpecialCharTok{+}\NormalTok{ theme.no\_axis }\SpecialCharTok{+}
    \FunctionTok{theme}\NormalTok{(}\AttributeTok{panel.grid.minor.y =} \FunctionTok{element\_blank}\NormalTok{()),}
\NormalTok{  df\_leads\_daily\_filtered }\SpecialCharTok{\%\textgreater{}\%}
    \FunctionTok{ggplot}\NormalTok{(}\FunctionTok{aes}\NormalTok{(mean)) }\SpecialCharTok{+}
    \FunctionTok{coord\_cartesian}\NormalTok{(}\AttributeTok{xlim =} \FunctionTok{c}\NormalTok{(}\SpecialCharTok{{-}}\FloatTok{0.02}\NormalTok{, }\FloatTok{0.4}\NormalTok{)) }\SpecialCharTok{+}
    \FunctionTok{geom\_boxplot}\NormalTok{(}\AttributeTok{color =} \FunctionTok{adjustcolor}\NormalTok{(}\StringTok{"royalblue"}\NormalTok{, }\AttributeTok{alpha.f =} \FloatTok{0.8}\NormalTok{), }\AttributeTok{fill =} \StringTok{"royalblue"}\NormalTok{, }\AttributeTok{alpha =} \FloatTok{0.5}\NormalTok{) }\SpecialCharTok{+}
    \FunctionTok{labs}\NormalTok{(}\AttributeTok{title =} \StringTok{""}\NormalTok{,}
         \AttributeTok{x =} \StringTok{""}\NormalTok{,}
         \AttributeTok{y =} \StringTok{""}\NormalTok{) }\SpecialCharTok{+}
\NormalTok{    theme.base }\SpecialCharTok{+}\NormalTok{ theme.no\_legend }\SpecialCharTok{+}\NormalTok{ theme.no\_axis }\SpecialCharTok{+}
    \FunctionTok{theme}\NormalTok{(}\AttributeTok{axis.text.y =} \FunctionTok{element\_blank}\NormalTok{(),}
          \AttributeTok{axis.ticks.y =} \FunctionTok{element\_blank}\NormalTok{(),}
          \AttributeTok{panel.grid.major.y =} \FunctionTok{element\_blank}\NormalTok{(),}
          \AttributeTok{panel.grid.minor.y =} \FunctionTok{element\_blank}\NormalTok{()),}
  \AttributeTok{nrow =} \DecValTok{3}\NormalTok{,}
  \AttributeTok{heights =} \FunctionTok{c}\NormalTok{(}\DecValTok{3}\NormalTok{, }\DecValTok{2}\NormalTok{, }\FloatTok{1.5}\NormalTok{)}
\NormalTok{)}
\end{Highlighting}
\end{Shaded}

\begin{center} \ifthenelse{ \equal{width=\textwidth,height=\textheight,keepaspectratio=true}{} } { \includegraphics[width=2.5cm,height=2.5cm,keepaspectratio=true]{tamanho-amostra_files/figure-latex/unnamed-chunk-10-1} } { \includegraphics[width=\textwidth,height=\textheight,keepaspectratio=true]{tamanho-amostra_files/figure-latex/unnamed-chunk-10-1} }  \end{center}

\begin{Shaded}
\begin{Highlighting}[]
\NormalTok{leads\_daily\_mean }\OtherTok{\textless{}{-}} \FunctionTok{mean}\NormalTok{(df\_leads\_complete\_filtered}\SpecialCharTok{$}\NormalTok{leads)}
\NormalTok{leads\_daily\_sd }\OtherTok{\textless{}{-}} \FunctionTok{sd}\NormalTok{(df\_leads\_complete\_filtered}\SpecialCharTok{$}\NormalTok{leads)}

\NormalTok{leads\_mean\_of\_means }\OtherTok{\textless{}{-}} \FunctionTok{mean}\NormalTok{(df\_leads\_daily\_filtered}\SpecialCharTok{$}\NormalTok{mean)}
\NormalTok{leads\_sd\_of\_means }\OtherTok{\textless{}{-}} \FunctionTok{sd}\NormalTok{(df\_leads\_daily\_filtered}\SpecialCharTok{$}\NormalTok{mean)}
\NormalTok{mean\_active\_listings }\OtherTok{\textless{}{-}} \FunctionTok{ceiling}\NormalTok{(}\FunctionTok{mean}\NormalTok{(df\_leads\_daily\_filtered}\SpecialCharTok{$}\NormalTok{active\_listings))}

\FunctionTok{sumtable}\NormalTok{(df\_leads\_daily\_filtered, }\AttributeTok{add.median =}\NormalTok{ T, }\AttributeTok{title =} \StringTok{"Média de leads por dia"}\NormalTok{)}
\end{Highlighting}
\end{Shaded}

\begin{table}

\caption{\label{tab:unnamed-chunk-11}Média de leads por dia}
\centering
\begin{tabular}[t]{lllllllll}
\toprule
Variable & N & Mean & Std. Dev. & Min & Pctl. 25 & Pctl. 50 & Pctl. 75 & Max\\
\midrule
mean & 171 & 0.12 & 0.069 & 0 & 0.068 & 0.11 & 0.16 & 0.35\\
total\_leads & 171 & 7.8 & 5.1 & 0 & 4 & 7 & 10 & 28\\
active\_listings & 171 & 66 & 14 & 25 & 58 & 71 & 76 & 93\\
\bottomrule
\end{tabular}
\end{table}

\subsection{Resultados}\label{resultados}

Assim, apesar dos pesares, temos uma média de
\texttt{\textasciitilde{}0.118} leads por dia, com um desvio padrão de
\texttt{\textasciitilde{}0.337}. Além disso, a média das médias diárias
é de \texttt{\textasciitilde{}0.118} com um desvio padrão de
\texttt{\textasciitilde{}0.069}.

\subsubsection{Tamanho da amostra}\label{tamanho-da-amostra}

Calculamos que o tamanho da amostra, a partir da equação

\[
\begin{aligned}
n' &\ge \left(Z_{\alpha / 2} \frac{\sigma}{\text{e}}\right)^2 \\
n  &= n' \cdot \frac{N-n}{N-1}
\end{aligned}
\]

onde, \(Z_{\alpha / 2}\) é o valor crítico da distribuição normal,
\(\text{e}\) é a margem de erro, \(\sigma\) é o desvio padrão, \(N\) é o
tamanho da população, \(n'\) é o tamanho da amostra com amostra
aleatória simples com reposição (AASc), e, \(n\) é o tamanho da amostra
sem reposição (AASs).

\begin{Shaded}
\begin{Highlighting}[]
\NormalTok{confidence }\OtherTok{\textless{}{-}} \FloatTok{0.8}
\NormalTok{z }\OtherTok{\textless{}{-}} \FunctionTok{abs}\NormalTok{(}\FunctionTok{qnorm}\NormalTok{((}\DecValTok{1} \SpecialCharTok{{-}}\NormalTok{ confidence) }\SpecialCharTok{/} \DecValTok{2}\NormalTok{))}
\NormalTok{e }\OtherTok{\textless{}{-}} \FloatTok{0.05}
\NormalTok{sigma }\OtherTok{\textless{}{-}}\NormalTok{ leads\_sd\_of\_means}
\NormalTok{size\_population }\OtherTok{\textless{}{-}}\NormalTok{ mean\_active\_listings}

\NormalTok{computed\_sample\_size\_aasc }\OtherTok{\textless{}{-}}\NormalTok{ (z }\SpecialCharTok{*}\NormalTok{ sigma }\SpecialCharTok{/}\NormalTok{ e)}\SpecialCharTok{\^{}}\DecValTok{2}
\NormalTok{computed\_sample\_size }\OtherTok{\textless{}{-}} \FunctionTok{ceiling}\NormalTok{(}
\NormalTok{  computed\_sample\_size\_aasc }\SpecialCharTok{*}
\NormalTok{    (size\_population }\SpecialCharTok{{-}}\NormalTok{ computed\_sample\_size\_aasc) }\SpecialCharTok{/}\NormalTok{ (size\_population }\SpecialCharTok{{-}} \DecValTok{1}\NormalTok{)}
\NormalTok{)}
\end{Highlighting}
\end{Shaded}

Lembrando que queremos estimar a média de leads diários, portanto, vamos
usar a média das médias diárias.

\textbf{Ta-dá!} Para obter uma margem de erro de \texttt{0.05} com
\texttt{80\%} de confiança, utilizando AASs, precisamos de uma amostra
de \texttt{4} anúncios.

\includegraphics{./images/flork-orgulhoso.png}

\subsubsection{Cálculo amostral}\label{cuxe1lculo-amostral}

\paragraph{Tamanho da amostra:
n\_calculado}\label{tamanho-da-amostra-n_calculado}

\begin{Shaded}
\begin{Highlighting}[]
\NormalTok{nivel\_conf }\OtherTok{\textless{}{-}} \FloatTok{0.80}
\NormalTok{z }\OtherTok{\textless{}{-}} \FunctionTok{qnorm}\NormalTok{(}\DecValTok{1} \SpecialCharTok{{-}}\NormalTok{ (}\DecValTok{1} \SpecialCharTok{{-}}\NormalTok{ nivel\_conf) }\SpecialCharTok{/} \DecValTok{2}\NormalTok{)}

\NormalTok{n\_calculado }\OtherTok{\textless{}{-}} \FunctionTok{ceiling}\NormalTok{(z}\SpecialCharTok{\^{}}\DecValTok{2} \SpecialCharTok{*}\NormalTok{ (}\FunctionTok{var}\NormalTok{(df\_leads\_daily\_filtered}\SpecialCharTok{$}\NormalTok{mean)}\SpecialCharTok{/}\FloatTok{0.05}\SpecialCharTok{\^{}}\DecValTok{2}\NormalTok{))}
\FunctionTok{print}\NormalTok{(}\FunctionTok{paste}\NormalTok{(}\StringTok{"n calculado:"}\NormalTok{, n\_calculado))}
\end{Highlighting}
\end{Shaded}

\begin{verbatim}
## [1] "n calculado: 4"
\end{verbatim}

\paragraph{Margem de erro calculada:
margem\_erro}\label{margem-de-erro-calculada-margem_erro}

\begin{Shaded}
\begin{Highlighting}[]
\NormalTok{margem\_erro }\OtherTok{\textless{}{-}}\NormalTok{ z }\SpecialCharTok{*}\NormalTok{ (}\FunctionTok{sd}\NormalTok{(df\_leads\_daily\_filtered}\SpecialCharTok{$}\NormalTok{mean) }\SpecialCharTok{/} \FunctionTok{sqrt}\NormalTok{(n\_calculado))}
\FunctionTok{print}\NormalTok{(}\FunctionTok{paste}\NormalTok{(}\StringTok{"margem de erro:"}\NormalTok{, margem\_erro))}
\end{Highlighting}
\end{Shaded}

\begin{verbatim}
## [1] "margem de erro: 0.0441555477843345"
\end{verbatim}

\paragraph{Geração de 1000 amostras aleatórias das médias diárias:
values\_mean\_amostral}\label{gerauxe7uxe3o-de-1000-amostras-aleatuxf3rias-das-muxe9dias-diuxe1rias-values_mean_amostral}

\begin{Shaded}
\begin{Highlighting}[]
\NormalTok{num\_amostras }\OtherTok{\textless{}{-}} \DecValTok{1000}

\FunctionTok{set.seed}\NormalTok{(}\DecValTok{2024}\NormalTok{)  }\CommentTok{\# Definida uma semente}
\NormalTok{values\_mean\_amostral }\OtherTok{\textless{}{-}} \FunctionTok{replicate}\NormalTok{(num\_amostras, \{}
\NormalTok{  amostra }\OtherTok{\textless{}{-}} \FunctionTok{sample}\NormalTok{(df\_leads\_daily\_filtered}\SpecialCharTok{$}\NormalTok{mean, n\_calculado, }\AttributeTok{replace =} \ConstantTok{FALSE}\NormalTok{)}
  \FunctionTok{mean}\NormalTok{(amostra)}
\NormalTok{\})}
\FunctionTok{print}\NormalTok{(values\_mean\_amostral)}
\end{Highlighting}
\end{Shaded}

\begin{verbatim}
##    [1] 0.11501045 0.12053098 0.11109110 0.10297871 0.12457675 0.08301267
##    [7] 0.13621481 0.09576870 0.17370370 0.05480625 0.15076811 0.11822179
##   [13] 0.06973701 0.21897062 0.11221218 0.12055298 0.09572781 0.14682857
##   [19] 0.09995705 0.05851051 0.12456044 0.09026366 0.17660294 0.09737654
##   [25] 0.09026039 0.09151259 0.08406773 0.15210757 0.11549145 0.13209376
##   [31] 0.18579282 0.10720585 0.08188196 0.17375969 0.16037594 0.08381849
##   [37] 0.18920420 0.09575427 0.14148180 0.11331197 0.05615262 0.14479614
##   [43] 0.14839135 0.05439013 0.14710229 0.04464182 0.10130275 0.15582309
##   [49] 0.16037055 0.13612197 0.10857065 0.19779777 0.10495531 0.08819991
##   [55] 0.13716130 0.11548992 0.12026925 0.08177614 0.12539557 0.09690003
##   [61] 0.06253743 0.13316670 0.10437695 0.16655359 0.09011701 0.16318125
##   [67] 0.07541667 0.09416823 0.07755831 0.09966420 0.21141645 0.12409310
##   [73] 0.13459221 0.09739927 0.07819444 0.22261667 0.14709124 0.09249035
##   [79] 0.14274059 0.14664408 0.10412306 0.12875817 0.12683412 0.15698306
##   [85] 0.13361571 0.14767070 0.10691453 0.18010415 0.07505994 0.07365862
##   [91] 0.05563862 0.15492588 0.10071009 0.12231382 0.18307334 0.16796136
##   [97] 0.16924225 0.11252412 0.13741518 0.10899131 0.05171785 0.12459350
##  [103] 0.11973539 0.14250284 0.09655303 0.19795997 0.13285505 0.10289261
##  [109] 0.08187977 0.15841841 0.11650789 0.17134175 0.08461952 0.08333333
##  [115] 0.07988202 0.10219928 0.11430107 0.14200924 0.09865689 0.10524944
##  [121] 0.11056171 0.13017590 0.13263823 0.09266368 0.13745556 0.13041105
##  [127] 0.13313928 0.15645058 0.08257447 0.11536797 0.09000323 0.12441300
##  [133] 0.10834608 0.10457235 0.15706415 0.09617105 0.13937387 0.12318490
##  [139] 0.15682265 0.09489734 0.11882562 0.09156508 0.09849630 0.15600768
##  [145] 0.09624745 0.08085345 0.17345467 0.09683952 0.11124562 0.14173692
##  [151] 0.11697149 0.12240599 0.09277280 0.13961578 0.10461560 0.09075611
##  [157] 0.09492356 0.17311064 0.06788427 0.09037399 0.09075946 0.18313253
##  [163] 0.14648089 0.11314233 0.11127471 0.12441335 0.10963490 0.15865297
##  [169] 0.13987781 0.09816841 0.09066198 0.14879102 0.11585039 0.14239148
##  [175] 0.12675835 0.15667129 0.12034901 0.07786546 0.10691743 0.17233643
##  [181] 0.09082228 0.05651069 0.11724354 0.15231536 0.15344379 0.14997163
##  [187] 0.13310505 0.16690780 0.16489638 0.13795081 0.15371224 0.13013116
##  [193] 0.06132624 0.16508621 0.13232323 0.09575545 0.13001565 0.14185455
##  [199] 0.07380289 0.06417078 0.11956225 0.15088722 0.16705912 0.14473253
##  [205] 0.11677991 0.18878441 0.08046140 0.19508785 0.07719941 0.13261991
##  [211] 0.12192550 0.14404078 0.16157088 0.10997336 0.13750038 0.08474427
##  [217] 0.09083762 0.19018290 0.08398325 0.14303418 0.14507659 0.14932372
##  [223] 0.07750367 0.18224437 0.10746407 0.12851424 0.10737435 0.16592290
##  [229] 0.12264137 0.11156851 0.12006108 0.12942590 0.08645317 0.13015900
##  [235] 0.08879254 0.11896437 0.11273119 0.10369417 0.11182439 0.12057211
##  [241] 0.10755890 0.12051264 0.12023252 0.15963686 0.13971196 0.15125487
##  [247] 0.08972557 0.08070175 0.12901310 0.10800067 0.13612237 0.08487927
##  [253] 0.13215812 0.09498067 0.10086874 0.12373715 0.16255768 0.09511816
##  [259] 0.16363940 0.14820529 0.13481801 0.06143332 0.14618352 0.14005666
##  [265] 0.13562500 0.10142517 0.10441521 0.14038542 0.15070654 0.14718663
##  [271] 0.09490923 0.08166202 0.11766585 0.10852093 0.13195011 0.06328730
##  [277] 0.16666748 0.14602155 0.16296794 0.08755647 0.12424623 0.07444876
##  [283] 0.18840290 0.14744230 0.12193028 0.06329427 0.09174012 0.10316492
##  [289] 0.17066579 0.12185497 0.09763171 0.06340713 0.09680062 0.18242012
##  [295] 0.09534100 0.14019413 0.10561469 0.14673765 0.13266720 0.18131070
##  [301] 0.09422644 0.11325989 0.16053879 0.13781073 0.12500364 0.06608991
##  [307] 0.12181467 0.11600862 0.10805478 0.10168172 0.12492159 0.11600112
##  [313] 0.07551317 0.06502223 0.09110823 0.14861284 0.09409332 0.08446886
##  [319] 0.15052464 0.11438307 0.11164044 0.13806324 0.15724105 0.08754098
##  [325] 0.10921573 0.09541464 0.09030974 0.07268261 0.12670455 0.15101090
##  [331] 0.08801803 0.14760530 0.16776896 0.15537187 0.13351153 0.16944662
##  [337] 0.07325878 0.15153017 0.12656988 0.12716106 0.09108394 0.10767123
##  [343] 0.16199713 0.09255574 0.09576167 0.15394901 0.09075741 0.21564013
##  [349] 0.12001959 0.08168803 0.06384389 0.14140620 0.08022774 0.06444327
##  [355] 0.10287677 0.07461527 0.09864820 0.15056130 0.06143856 0.17319352
##  [361] 0.07398140 0.08839333 0.18788972 0.09120520 0.13487847 0.07534088
##  [367] 0.06312955 0.12705378 0.11028139 0.07062420 0.12800726 0.07826742
##  [373] 0.16466256 0.06018227 0.08882037 0.15357952 0.11770299 0.20127650
##  [379] 0.09568204 0.07494172 0.10467095 0.13306524 0.11785317 0.12359634
##  [385] 0.18304318 0.14135985 0.17651738 0.10706933 0.13154745 0.19612937
##  [391] 0.11137750 0.07828773 0.11560231 0.15437383 0.09643543 0.07669388
##  [397] 0.07369041 0.14024758 0.18444767 0.08064707 0.08952425 0.16353727
##  [403] 0.09845417 0.09351054 0.15303017 0.06666223 0.13253221 0.10224385
##  [409] 0.09491994 0.12094928 0.14838051 0.10516654 0.12866701 0.06028531
##  [415] 0.07571568 0.09414094 0.09963901 0.15265755 0.11413845 0.14633291
##  [421] 0.08928612 0.10712758 0.10956925 0.08517599 0.11509770 0.11686368
##  [427] 0.11803050 0.15401210 0.13935227 0.07793311 0.12998391 0.21403708
##  [433] 0.14637662 0.11751408 0.08850310 0.10934700 0.10652078 0.13275141
##  [439] 0.12726447 0.16197786 0.14539928 0.14369534 0.07195517 0.14513180
##  [445] 0.16041293 0.12487871 0.09424433 0.10290916 0.12458333 0.14246529
##  [451] 0.09116191 0.06619709 0.13165001 0.06580951 0.05959206 0.09439066
##  [457] 0.10431267 0.09549315 0.14546203 0.15633667 0.15551851 0.06599261
##  [463] 0.13834207 0.09745005 0.11628013 0.13204225 0.09065799 0.11164104
##  [469] 0.19199029 0.10760784 0.16878900 0.11065949 0.07152330 0.10753159
##  [475] 0.19462438 0.15285131 0.11219450 0.16493657 0.17159894 0.09994690
##  [481] 0.12420271 0.10314597 0.15541167 0.15784660 0.05433068 0.10642019
##  [487] 0.12885383 0.09672592 0.14524544 0.12337382 0.12606127 0.09668833
##  [493] 0.15114182 0.04237041 0.13549306 0.09411039 0.18726221 0.13444767
##  [499] 0.11743742 0.16192968 0.11376408 0.11933795 0.09629542 0.12896586
##  [505] 0.12219267 0.14217301 0.11489622 0.17512582 0.11685685 0.08552226
##  [511] 0.20249704 0.10588134 0.10449846 0.10213375 0.16693913 0.17752208
##  [517] 0.08568489 0.11397715 0.07364404 0.08783926 0.11869374 0.06462019
##  [523] 0.12151683 0.13323876 0.13834767 0.08520906 0.05902762 0.09064746
##  [529] 0.10359856 0.12251711 0.10352958 0.13130342 0.12640096 0.13483767
##  [535] 0.15160256 0.16458560 0.16276178 0.14242274 0.07653356 0.12497071
##  [541] 0.12991099 0.11283061 0.06321590 0.11406861 0.15163692 0.11470624
##  [547] 0.10718524 0.05871578 0.15156532 0.07709132 0.15397564 0.13153010
##  [553] 0.11565936 0.11040206 0.07669039 0.11068129 0.18954988 0.10252352
##  [559] 0.11154099 0.05782496 0.08524272 0.10593021 0.18473127 0.09471143
##  [565] 0.10435739 0.13582725 0.07702156 0.17527541 0.11679022 0.10984573
##  [571] 0.12067373 0.06744201 0.11875618 0.09728205 0.12798122 0.12730242
##  [577] 0.16965029 0.13288345 0.09112676 0.08762969 0.16085706 0.15357972
##  [583] 0.08802810 0.12121828 0.15746212 0.09572978 0.20961277 0.11052632
##  [589] 0.13497380 0.11259923 0.08522472 0.11273006 0.12462760 0.18238880
##  [595] 0.10750000 0.14478218 0.08144231 0.14484548 0.10905679 0.14349399
##  [601] 0.09389817 0.19483871 0.12862008 0.11205189 0.10506490 0.14397556
##  [607] 0.07611922 0.18773228 0.08389578 0.12023705 0.11112773 0.13134034
##  [613] 0.07239113 0.16736180 0.13345738 0.10314285 0.13272500 0.11416817
##  [619] 0.08257859 0.08978187 0.09032308 0.14382845 0.13655819 0.12243121
##  [625] 0.13307020 0.09884796 0.13197090 0.11699630 0.08393689 0.11234698
##  [631] 0.08320202 0.09739503 0.08851751 0.13042771 0.13424974 0.10636523
##  [637] 0.10896307 0.09434831 0.10737703 0.09841115 0.12072168 0.05854226
##  [643] 0.12133313 0.13698718 0.17621255 0.10181923 0.08299623 0.16296748
##  [649] 0.18918089 0.15682856 0.11544827 0.14323309 0.21024887 0.11457024
##  [655] 0.11639675 0.10411297 0.09346921 0.08876689 0.12969704 0.09921662
##  [661] 0.13106053 0.08763889 0.10833333 0.12685364 0.13740153 0.15813783
##  [667] 0.11921332 0.03650976 0.12774600 0.10686078 0.12261613 0.14403362
##  [673] 0.11256260 0.15952336 0.11876910 0.10632060 0.10300039 0.10713472
##  [679] 0.09885644 0.11552254 0.15342163 0.14623273 0.14755866 0.11297365
##  [685] 0.10210409 0.14046654 0.05859516 0.10834175 0.07750883 0.11015389
##  [691] 0.14006763 0.11351659 0.13531130 0.16961175 0.16557118 0.15861565
##  [697] 0.07440031 0.11826754 0.12473361 0.11327543 0.13320431 0.07247569
##  [703] 0.10741490 0.07639196 0.11170625 0.14370208 0.15531478 0.18051708
##  [709] 0.15342593 0.11525129 0.13023596 0.08626489 0.09572467 0.12201919
##  [715] 0.11810581 0.10947978 0.08460341 0.06779892 0.09362751 0.19387042
##  [721] 0.10944277 0.13986545 0.13345868 0.12800300 0.08983650 0.11042604
##  [727] 0.06485405 0.11801663 0.18409136 0.05220509 0.15963701 0.15706503
##  [733] 0.07190155 0.12243411 0.14874100 0.08541763 0.12105263 0.13378995
##  [739] 0.08721195 0.08465808 0.12305963 0.03665612 0.13650677 0.10855024
##  [745] 0.11588731 0.07204548 0.10455856 0.08623450 0.19206615 0.11555556
##  [751] 0.10817166 0.08591270 0.11328157 0.12033419 0.16419774 0.09963933
##  [757] 0.07278327 0.08672099 0.11936407 0.18561624 0.09658862 0.15492693
##  [763] 0.11783642 0.09436067 0.15880127 0.11734139 0.11607920 0.12083572
##  [769] 0.13464093 0.07321319 0.06897396 0.13167497 0.12261674 0.18598808
##  [775] 0.08766192 0.14520982 0.14188312 0.11129184 0.10070849 0.13973793
##  [781] 0.14473683 0.04293567 0.16352644 0.12053357 0.13136324 0.09109015
##  [787] 0.14949309 0.15028853 0.07275531 0.10873192 0.10353020 0.15067862
##  [793] 0.13295997 0.09903794 0.12126367 0.10619642 0.11577551 0.09741416
##  [799] 0.10303083 0.16336209 0.07468448 0.12688672 0.20067718 0.09438001
##  [805] 0.13411645 0.09800125 0.12003978 0.11924435 0.16617063 0.10460820
##  [811] 0.07899351 0.13074128 0.19100325 0.12034989 0.15406506 0.11311666
##  [817] 0.10948276 0.17641148 0.13147416 0.07137629 0.07157264 0.13987170
##  [823] 0.11817033 0.07022862 0.09065518 0.08655112 0.11153726 0.09377354
##  [829] 0.15348238 0.14541270 0.11279333 0.08029902 0.10988914 0.09864748
##  [835] 0.04908732 0.12400598 0.11483336 0.07111494 0.13699660 0.15038246
##  [841] 0.06788227 0.10735348 0.04728320 0.11848263 0.16274482 0.19712644
##  [847] 0.15891332 0.15566432 0.14326144 0.07982504 0.09615245 0.20403481
##  [853] 0.15933310 0.15677834 0.10596646 0.13324786 0.10344433 0.10080897
##  [859] 0.11032091 0.13895458 0.15151250 0.13212747 0.09568897 0.11958796
##  [865] 0.09232497 0.17784743 0.10350735 0.15174085 0.13357947 0.11253725
##  [871] 0.11099545 0.15021380 0.12243308 0.07417115 0.17341620 0.13148381
##  [877] 0.15066770 0.14064624 0.11848404 0.11375182 0.11698614 0.11072613
##  [883] 0.17195626 0.10726689 0.07839983 0.11812615 0.08122367 0.07946451
##  [889] 0.11110739 0.13470548 0.10818630 0.09087566 0.09871795 0.15650427
##  [895] 0.09196237 0.12924976 0.10883028 0.07098787 0.14520952 0.17086172
##  [901] 0.07636307 0.12141721 0.07170370 0.18029423 0.06057971 0.14455351
##  [907] 0.15500743 0.09768901 0.10056610 0.16145421 0.14629122 0.17222244
##  [913] 0.15901300 0.07292711 0.13571886 0.14330383 0.09477174 0.17895336
##  [919] 0.13061900 0.11766351 0.09324853 0.17211165 0.14759512 0.10463040
##  [925] 0.12562443 0.09895584 0.10279001 0.09292364 0.13075414 0.14173815
##  [931] 0.18028221 0.19474099 0.13085720 0.15646831 0.12495450 0.12288387
##  [937] 0.12247725 0.11607622 0.14358858 0.11321646 0.08727453 0.08809829
##  [943] 0.14305634 0.12203796 0.16651556 0.10043986 0.14905618 0.15694084
##  [949] 0.10592109 0.09173419 0.10444314 0.11203691 0.11441911 0.05546775
##  [955] 0.06946883 0.11278551 0.16946171 0.08301211 0.09921769 0.15009778
##  [961] 0.14001390 0.11035546 0.17359703 0.11968432 0.11328921 0.16894285
##  [967] 0.08287174 0.14719281 0.12183009 0.10592395 0.11273211 0.09451588
##  [973] 0.07743228 0.11645019 0.12862903 0.12802290 0.11093449 0.09666231
##  [979] 0.12612719 0.16818479 0.09892972 0.18486479 0.08561368 0.16581661
##  [985] 0.11188554 0.12999445 0.12659004 0.07324589 0.14778935 0.14147065
##  [991] 0.13944064 0.07943051 0.11850916 0.14996738 0.08294417 0.14984653
##  [997] 0.13354798 0.16512213 0.11628404 0.06127845
\end{verbatim}

\paragraph{Média amostral das 1000 amostras obtidas:
mean\_amostral}\label{muxe9dia-amostral-das-1000-amostras-obtidas-mean_amostral}

\begin{Shaded}
\begin{Highlighting}[]
\CommentTok{\# Média das 1000 amostras geradas de tamanho n\_calculado para estimar a média populacional}

\NormalTok{mean\_amostral }\OtherTok{\textless{}{-}} \FunctionTok{mean}\NormalTok{(values\_mean\_amostral)}
\FunctionTok{print}\NormalTok{(}\FunctionTok{paste}\NormalTok{(}\StringTok{"Média amostral:"}\NormalTok{, mean\_amostral))}
\end{Highlighting}
\end{Shaded}

\begin{verbatim}
## [1] "Média amostral: 0.119999608259177"
\end{verbatim}

\paragraph{Cálculo do erro das médias amostrais obtidas:
erro}\label{cuxe1lculo-do-erro-das-muxe9dias-amostrais-obtidas-erro}

\begin{Shaded}
\begin{Highlighting}[]
\CommentTok{\# Calculando o maior e o menor valor das médias amostrais}
\NormalTok{highest\_media }\OtherTok{\textless{}{-}} \FunctionTok{max}\NormalTok{(values\_mean\_amostral)}
\NormalTok{lowest\_media }\OtherTok{\textless{}{-}} \FunctionTok{min}\NormalTok{(values\_mean\_amostral)}

\CommentTok{\# Calculando o erro (diferença entre o maior e o menor valor)}
\NormalTok{erro }\OtherTok{\textless{}{-}}\NormalTok{ highest\_media }\SpecialCharTok{{-}}\NormalTok{ lowest\_media}

\CommentTok{\# Exibindo os resultados}
\FunctionTok{print}\NormalTok{(}\FunctionTok{paste}\NormalTok{(}\StringTok{"Maior média amostral:"}\NormalTok{, highest\_media))}
\end{Highlighting}
\end{Shaded}

\begin{verbatim}
## [1] "Maior média amostral: 0.222616666012892"
\end{verbatim}

\begin{Shaded}
\begin{Highlighting}[]
\FunctionTok{print}\NormalTok{(}\FunctionTok{paste}\NormalTok{(}\StringTok{"Menor média amostral:"}\NormalTok{, lowest\_media))}
\end{Highlighting}
\end{Shaded}

\begin{verbatim}
## [1] "Menor média amostral: 0.0365097630920416"
\end{verbatim}

\begin{Shaded}
\begin{Highlighting}[]
\FunctionTok{print}\NormalTok{(}\FunctionTok{paste}\NormalTok{(}\StringTok{"Erro (diferença entre maior e menor média):"}\NormalTok{, erro))}
\end{Highlighting}
\end{Shaded}

\begin{verbatim}
## [1] "Erro (diferença entre maior e menor média): 0.186106902920851"
\end{verbatim}

\subsection{Sugestão de tamanho da
amostra}\label{sugestuxe3o-de-tamanho-da-amostra}

Para trabalhos futuros sugerimos que o tamanho da amostra seja de

\end{document}
